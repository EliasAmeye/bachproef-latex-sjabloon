%%=============================================================================
%% Pgpool-II
%%=============================================================================

\chapter{Pgpool-II}
\label{ch:Pgpool-II}

In dit hoofdstuk wordt meer informatie verschaft over Pgpool-II als een PostgreSQL High Availability cluster oplossing. Er zal worden ingegaan op de verschillende requirements, waaraan voldaan moet worden. Hierbij wordt dan telkens ook een score gegeven die in Hoofdstuk~\ref{ch:Verwerking resultaten}: Verwerking resultaten verder geanalyseerd zullen worden.

\section{\IfLanguageName{dutch}{Inleiding tot Pgpool-II}{Inleiding tot Pgpool-II}}
\label{sec:Inleiding tot Pgpool-II}

\section{\IfLanguageName{dutch}{Requirements}{Requirements}}
\label{sec:Requirements}

\subsection{\IfLanguageName{dutch}{Must have}{Must have}}
\label{subsec:Must have}

%Pgpool-II kan aan de hand van Watchdog aan de vooropgestelde “Must haves” voldoen. Pgpool-II krijgt hierdoor 10/15 punten.

\subsubsection{\IfLanguageName{dutch}{Replicatie}{Replicatie}}
\label{subsubsec:Replicatie}

Pgpool-II beschikt over een tool, genaamd Watchdog, die een subproces is van Pgpool-II dat dient om High Availability toe te voegen. Watchdog wordt gebruikt om single points of failure op te lossen door meerdere Pgpool-II nodes te coördineren. Het coördineert meerdere Pgpool-II nodes door informatie met elkaar uit te wisselen. Op die manier kan het zeker zijn dat de database service onaangetast blijft.

Standaard zal er opnieuw gebruik worden gemaakt van streaming replication bij Pgpool-II.

\subsubsection{\IfLanguageName{dutch}{Failover}{Failover}}
\label{subsubsec:Failover}

Pgpool-II voorziet ook automatische failover binnenin Watchdog. Pgpool-II kan ook andere soorten van failover configuraties aan zoals failover\_requires\_consensus of \newline failover\_when\_quorum\_exists. 

Bij Pgpool-II, wanneer het failover\_command geconfigureerd is, zal deze automatisch tijdens een failover worden uitgevoerd. Hierin zal een nieuwe primary node uit de bestaande standby nodes gekozen worden. Ook kan dit commando signalen uitsturen, zoals een mail om te laten weten dat er failover plaatsvindt binnen de cluster.

\subsubsection{\IfLanguageName{dutch}{Monitoring}{Monitoring}}
\label{subsubsec:Monitoring}

Watchdog kan ook remote de gezondheid controleren van de node waarop het is geïnstalleerd door de verbinding met upstream nodes te monitoren. Als de monitoring faalt, behandelt watchdog dit als het falen van een lokale Pgpool-II node en zal het de nodige acties ondernemen.

Pgpool-II kan meerdere nodes beheren en hierin replicatie opzetten om High Availability te verkrijgen

\subsection{\IfLanguageName{dutch}{Should have}{Should have}}
\label{subsec:Should have}

%Aan de hand hiervan krijgt Pgpool-II 4/5 punten voor “Should have”. Geen 5/5 omdat er (nog) geen update in 2021 is geweest.

\subsubsection{\IfLanguageName{dutch}{Actieve ondersteuning in 2020-2021}{Actieve ondersteuning in 2020-2021}}
\label{subsubsec:Actieve ondersteuning in 2020-2021}

De laatste release van pgpool-II was op 26 november 2020. Hieruit kan worden besloten dat Pgpool-II nog een geldige, bruikbare oplossing is.

\subsubsection{\IfLanguageName{dutch}{Open source}{Open source}}
\label{subsubsec:Open source}

Pgpool-II is een open source project. De source code wordt onderhouden door hun git-repository.

\subsection{\IfLanguageName{dutch}{Could have}{Could have}}
\label{subsec:Could have}


\subsubsection{\IfLanguageName{dutch}{Grafische interface}{Grafische interface}}
\label{subsubsec:Grafische interface}

Op eerste zicht is geen algemeen gekende grafische interface aanwezig voor Pgpool-II.

\subsubsection{\IfLanguageName{dutch}{Beperkte manuele interventie}{Beperkte manuele interventie}}
\label{subsubsec:Beperkte manuele interventie}

Via de tool Watchdog is het mogelijk om bepaalde elementen in Pgpool-II te automatiseren. Automatische failover is hier een voorbeeld van. Doch andere elementen zijn wel nog manueel te configureren.
