%%=============================================================================
%% Replication Manager (repmgr)
%%=============================================================================

\chapter{Replication Manager (repmgr)}
\label{ch:Replication Manager (repmgr)}

In dit hoofdstuk wordt meer informatie verschaft over Replication Manager (repmgr) als een PostgreSQL High Availability cluster oplossing. Er zal worden ingegaan op de verschillende requirements, waaraan voldaan moet worden. Hierbij wordt dan telkens ook een score gegeven die in Hoofdstuk~\ref{ch:Verwerking resultaten}: Verwerking resultaten verder geanalyseerd zullen worden.

\section{\IfLanguageName{dutch}{Inleiding tot Replication Manager (repmgr)}{Inleiding tot Replication Manager (repmgr)}}
\label{sec:Inleiding tot Replication Manager (repmgr)}

\section{\IfLanguageName{dutch}{Requirements}{Requirements}}
\label{sec:Requirements}

\subsection{\IfLanguageName{dutch}{Must have}{Must have}}
\label{subsec:Must have}

Replication Manager (RepMgr) is een oplossing ontwikkeld voor het beheren van replicatie en failover van PostgreSQL clusters. Het biedt de tools aan om replicatie van PostgreSQL op te zetten, te configureren, te beheren en te monitoren. Het laat ook toe om handmatige omschakeling en failover taken uit te voeren met behulp van repmgr utility. Dit is een gratis tool die ondersteuning en verbetering biedt van PostgreSQL's ingebouwde streaming replicatie.
Voorbeelden van tools zijn repmgr en repmgrd.
Replication Manager (RepMgr) biedt ook de tools om primary en standby nodes op te zetten, en om in geval van faalscenario automatische failover te doen.

De functionaliteiten van Replication Manager (RepMgr) sluiten voldoende aan op de vooropgestelde “Must haves” en krijgt hiervoor 12/15 punten.

\subsubsection{\IfLanguageName{dutch}{Replicatie}{Replicatie}}
\label{subsubsec:Replicatie}

\subsubsection{\IfLanguageName{dutch}{Failover}{Failover}}
\label{subsubsec:Failover}

\subsubsection{\IfLanguageName{dutch}{Monitoring}{Monitoring}}
\label{subsubsec:Monitoring}

\subsection{\IfLanguageName{dutch}{Should have}{Should have}}
\label{subsec:Should have}

Replication Manager (RepMgr) is open source, ontwikkeld door 2ndQuadrant.
De laatste release was op 22 oktober 2020.

Replication Manager (RepMgr) krijgt hierdoor een score van 4/5 voor “Should have”. Doordat er (nog) geen release is in 2021, krijgt het geen 5/5.

\subsubsection{\IfLanguageName{dutch}{Actieve ondersteuning in 2020-2021}{Actieve ondersteuning in 2020-2021}}
\label{subsubsec:Actieve ondersteuning in 2020-2021}

\subsubsection{\IfLanguageName{dutch}{Open source}{Open source}}
\label{subsubsec:Open source}

\subsection{\IfLanguageName{dutch}{Could have}{Could have}}
\label{subsec:Could have}

Op eerste zicht lijkt er niet direct een alomgekende grafische interface aanwezig te zijn voor Replication Manager (repmgr).

Bij Replication Manager (repmgr) is er nog veel dat manueel moet gedaan worden.

\subsubsection{\IfLanguageName{dutch}{Grafische interface}{Grafische interface}}
\label{subsubsec:Grafische interface}

\subsubsection{\IfLanguageName{dutch}{Beperkte manuele interventie}{Beperkte manuele interventie}}
\label{subsubsec:Beperkte manuele interventie}
