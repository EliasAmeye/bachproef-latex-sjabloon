%%=============================================================================
%% Methodologie
%%=============================================================================

\chapter{\IfLanguageName{dutch}{Methodologie}{Methodology}}
\label{ch:methodologie}

%% TODO: Hoe ben je te werk gegaan? Verdeel je onderzoek in grote fasen, en
%% licht in elke fase toe welke stappen je gevolgd hebt. Verantwoord waarom je
%% op deze manier te werk gegaan bent. Je moet kunnen aantonen dat je de best
%% mogelijke manier toegepast hebt om een antwoord te vinden op de
%% onderzoeksvraag.

Zoals in vele andere onderzoeken ook het geval is, is dit onderzoek gestart met een diepgaande en extensieve literatuurstudie over PostgreSQL, High Availability, Puppet en de reeds bestaande High Available PostgreSQL cluster oplossingen. Deze literatuurstudie is terug te vinden in Hoofdstuk 2: Stand van zaken.

Na de literatuurstudie zal worden geduid hoe de verschillende High Available PostgreSQL cluster oplossingen geschift zullen worden. In deze schifting zullen verschillende requirements opgezet worden waaraan de verschillende oplossingen zullen afgetoetst worden. Dit zal gebeuren aan de hand van een requirementsanalyse waarin de verschillende requirements aan bod zullen komen. Deze analyse zal van groot belang zijn in dit onderzoek aangezien ze de basis zullen vormen waarop we een High Available PostgreSQL cluster oplossing zullen beoordelen en kiezen.
Uit deze schifting, aan de hand van de requirements worden dan de beste twee kandidaten verkozen. %waarin verder zal verdiept worden.

Na de schifting en de verdieping van de twee High Available PostgreSQL cluster oplossingen zal er in dit onderzoek één oplossing gebruikt worden waar een proof of concept mee gemaakt zal worden. Hier zal dan, door gebruik te maken van Puppet de reproduceerbaarheid van de High Available PostgreSQL cluster zeer eenvoudig en snel moeten verlopen.

2. Opstelling schifting en keuze 2 oplossingen
% Bij de schifting is het belangrijk dat de oplossing open source is. Dat het een hedendaagse community heeft ( 2020 state-of-the-art) en de 3 punten waar Ruben waarden aan hechtte.


3. Vergelijkende studie van 2 oplossingen waarin ik deze 2 oplossing volledig uitleg en waarom deze juist gekozen zijn geweest.
% Hierin ga ik dan deze 2 oplossingen volledig toelichten en zeggen hoe zij juist in postgres oplossingen aanbieden op verschillende vlakken.


4. Keuze van 1 oplossing die ik zal voorleggen als ultieme oplossing, deze oplossing zal ook gebruikt worden voor de proof-of-concept
%Keuze oplossing uitleggen.

5. Proof-of-concept
%Aan de hand van foto's en uitleg de configuratie en installatie van de cluster omgeving uitleggen en dan van elke server zijn functionaliteiten uitleggen, en hoe deze in verbinding staat met de andere servers.


