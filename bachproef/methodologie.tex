%%=============================================================================
%% Methodologie
%%=============================================================================

\chapter{\IfLanguageName{dutch}{Methodologie}{Methodology}}
\label{ch:methodologie}

%% TODO: Hoe ben je te werk gegaan? Verdeel je onderzoek in grote fasen, en
%% licht in elke fase toe welke stappen je gevolgd hebt. Verantwoord waarom je
%% op deze manier te werk gegaan bent. Je moet kunnen aantonen dat je de best
%% mogelijke manier toegepast hebt om een antwoord te vinden op de
%% onderzoeksvraag.

1. Literatuurstudie

2. Opstelling schifting en keuze 2 oplossingen
% Bij de schifting is het belangrijk dat de oplossing open source is. Dat het een hedendaagse community heeft ( 2020 state-of-the-art) en de 3 punten waar Ruben waarden aan hechtte.


3. Vergelijkende studie van 2 oplossingen waarin ik deze 2 oplossing volledig uitleg en waarom deze juist gekozen zijn geweest.
% Hierin ga ik dan deze 2 oplossingen volledig toelichten en zeggen hoe zij juist in postgres oplossingen aanbieden op verschillende vlakken.


4. Keuze van 1 oplossing die ik zal voorleggen als ultieme oplossing, deze oplossing zal ook gebruikt worden voor de proof-of-concept
%Keuze oplossing uitleggen.

5. Proof-of-concept
%Aan de hand van foto's en uitleg de configuratie en installatie van de cluster omgeving uitleggen en dan van elke server zijn functionaliteiten uitleggen, en hoe deze in verbinding staat met de andere servers.
