%%=============================================================================
%% Methodologie
%%=============================================================================

\chapter{\IfLanguageName{dutch}{Methodologie}{Methodology}}
\label{ch:methodologie}

%% TODO: Hoe ben je te werk gegaan? Verdeel je onderzoek in grote fasen, en
%% licht in elke fase toe welke stappen je gevolgd hebt. Verantwoord waarom je
%% op deze manier te werk gegaan bent. Je moet kunnen aantonen dat je de best
%% mogelijke manier toegepast hebt om een antwoord te vinden op de
%% onderzoeksvraag.

Zoals in vele andere onderzoeken ook het geval is, is dit onderzoek gestart met een diepgaande en extensieve literatuurstudie over PostgreSQL, High Availability, Puppet en de reeds bestaande High Available PostgreSQL cluster oplossingen. Deze literatuurstudie is terug te vinden in Hoofdstuk~\ref{ch:stand-van-zaken} Stand van zaken.

Na de literatuurstudie zal worden geduid hoe de verschillende High Available PostgreSQL cluster oplossingen geschift zullen worden. In deze schifting zullen verschillende requirements opgezet worden waaraan de verschillende oplossingen zullen afgetoetst worden. Dit zal gebeuren aan de hand van een requirementsanalyse waarin de verschillende requirements aan bod zullen komen. Deze requirements zullen dan aan de hand van de MoSCoW-methode geprioriteerd worden om zo het belang van elke requirement in te schatten. Deze analyse zal van groot belang zijn in dit onderzoek aangezien ze de basis zullen vormen waarop we een High Available PostgreSQL cluster oplossing zullen beoordelen en kiezen.
Uit deze schifting, aan de hand van de requirements worden dan de beste twee kandidaten verkozen die in Hoofdstuk~\ref{ch:Oplossing 1: Patroni} Oplossing 1: Patroni en in Hoofdstuk~\ref{ch:Oplossing 2: PostgreSQL Automatic Failover (PAF)} Oplossing 2: PostgreSQL Automatic Failover (PAF) verder verdiept zullen worden. In deze twee hoofdstukken worden de functionele requirements of de must have's nog eens doorlopen en kan er al in detail getreden worden over het gebruik van bepaalde tools of code.


Na de schifting en de verdieping van de twee High Available PostgreSQL cluster oplossingen zal er in dit onderzoek één oplossing gebruikt worden waar een proof of concept mee gemaakt zal worden. Hier zal dan gekeken worden om de verschillende functionele requirements te implementeren in de cluster. Op deze manier zal dan worden aangetoond dat deze oplossing een volwaardig PostgreSQL High Availability cluster oplossing is voor bedrijven om te gebruiken.%, door gebruik te maken van Puppet de reproduceerbaarheid van de High Available PostgreSQL cluster zeer eenvoudig en snel moeten verlopen.

Uiteindelijk zal dit onderzoek eindigen met een conclusie die gevormd wordt uit het gevoerde onderzoek.

