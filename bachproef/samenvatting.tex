%%=============================================================================
%% Samenvatting
%%=============================================================================

% TODO: De "abstract" of samenvatting is een kernachtige (~ 1 blz. voor een
% thesis) synthese van het document.
%
% Deze aspecten moeten zeker aan bod komen:
% - Context: waarom is dit werk belangrijk?
% - Nood: waarom moest dit onderzocht worden?
% - Taak: wat heb je precies gedaan?
% - Object: wat staat in dit document geschreven?
% - Resultaat: wat was het resultaat?
% - Conclusie: wat is/zijn de belangrijkste conclusie(s)?
% - Perspectief: blijven er nog vragen open die in de toekomst nog kunnen
%    onderzocht worden? Wat is een mogelijk vervolg voor jouw onderzoek?
%
% LET OP! Een samenvatting is GEEN voorwoord!

%%---------- Nederlandse samenvatting -----------------------------------------
%
% TODO: Als je je bachelorproef in het Engels schrijft, moet je eerst een
% Nederlandse samenvatting invoegen. Haal daarvoor onderstaande code uit
% commentaar.
% Wie zijn bachelorproef in het Nederlands schrijft, kan dit negeren, de inhoud
% wordt niet in het document ingevoegd.

\IfLanguageName{english}{%
\selectlanguage{dutch}
\chapter*{Samenvatting}
\lipsum[1-4]
\selectlanguage{english}
}{}

%%---------- Samenvatting -----------------------------------------------------
% De samenvatting in de hoofdtaal van het document

\chapter*{\IfLanguageName{dutch}{Samenvatting}{Abstract}}

% Deze aspecten moeten zeker aan bod komen:
% - Context: waarom is dit werk belangrijk?
% - Nood: waarom moest dit onderzocht worden?
% - Taak: wat heb je precies gedaan?
% - Object: wat staat in dit document geschreven?
% - Resultaat: wat was het resultaat?
% - Conclusie: wat is/zijn de belangrijkste conclusie(s)?
% - Perspectief: blijven er nog vragen open die in de toekomst nog kunnen
%    onderzocht worden? Wat is een mogelijk vervolg voor jouw onderzoek?


%Dit onderzoek kan dienen als basis voor de keuze tussen verschillende PostgreSQL High Availability cluster oplossingen. Doordat PostgreSQL een zeer populair open source object-relationeel databank systeem is, en het gebruik er van blijft stijgen, is het zeker interessant om te kijken hoe een bedrijf in 2020-2021 zijn High Availability clusters kan updaten, of creëren. 
 

%De ultieme oplossing uit dit onderzoek blijkt Patroni te zijn die zeer goed aan de verschillende requirements voldoet. De tweede gekozen oplossing komt te liggen bij PostgreSQL Automatic Failover (PAF).

Dit onderzoek is belangrijk omdat het als houvast kan dienen bij het maken van keuzes bij de selectie van een PostgreSQL High Availability cluster oplossing. Deze verschillende oplossingen zullen in dit onderzoek aan bod komen. Het zal een idee geven van het huidige landschap van High Availability cluster oplossingen voor PostgreSQL.

Deze oplossingen worden aan de hand van verschillende requirements met elkaar vergeleken. Hieruit brengt dit onderzoek één oplossing voor die gebruikt kan worden om High Availability te implementeren. In dit onderzoek vind u een inleiding tot High Availability in een PostgreSQL cluster met hieraan verbonden de huidige stand van zaken. Hierna worden vier oplossingen met elkaar vergeleken aan de hand van vooropgestelde requirements. Aan de hand van deze requirements krijgt elke oplossing een score toegewezen waarna bepaald zal worden met welke oplossing verder gewerkt wordt voor de opzet van een proof of concept.

In dit onderzoek ligt de focus vooral op High Availability in een PostgreSQL omgeving. Er wordt gekeken naar verschillende oplossingen zoals Patroni, PostgreSQL Automatic Failover (PAF), Pgpool-II en Replication Manager (repmgr). Deze oplossingen zullen overlopen worden aan de hand van de requirements en zullen daarna geanalyseerd worden. Tot slot is er een proof of concept die zal duiden dat deze oplossing voldoet aan de requirements.

De twee oplossingen, met de hoogste score zullen als winnaar beschouwd worden en worden dan voorgelegd als geprefereerde High Availability cluster oplossingen voor PostgreSQL.

Dit onderzoek kan ook een voorzet zijn voor toekomstige soortgelijke onderzoeken, waar gekeken kan worden of de huidige oplossingen nog voldoen aan de verwachtingen over High Availability clustering in de toekomst.


%Deze vergelijkende studie kan een meerwaarde bieden aan bedrijven die werken met een kleine of grote PostgreSQL cluster waarin zij High Availability willen implementeren. Het kan ook een meerwaarde zijn voor bedrijven die al High Availability implementaties hebben in hun cluster, maar die een frisse blik nodig hebben, of willen upgraden naar een meer hedendaagse oplossing. Dit onderzoek richt zich bijgevolg op systeem- en netwerkbeheerders die dagelijks bezig zijn met het onderhouden en/of beheren van servers, meer specifiek database servers.