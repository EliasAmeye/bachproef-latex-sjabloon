%%=============================================================================
%% Voorwoord
%%=============================================================================

\chapter*{\IfLanguageName{dutch}{Woord vooraf}{Preface}}
\label{ch:voorwoord}

%% TODO:
%% Het voorwoord is het enige deel van de bachelorproef waar je vanuit je
%% eigen standpunt (``ik-vorm'') mag schrijven. Je kan hier bv. motiveren
%% waarom jij het onderwerp wil bespreken.
%% Vergeet ook niet te bedanken wie je geholpen/gesteund/... heeft
Deze bachelorproef werd geschreven in het kader van het voltooien van de opleiding Toegepaste Informatica afstudeerrichting Systeem- en Netwerkbeheer. Ik heb gekozen voor dit onderwerp omdat High Availability clustering mij tijdens de opleiding altijd al interesseerde.

In dit onderzoek zal ik het hebben over High Availability oplossingen en het belang ervan in PostgreSQL clusters. Met de Coronapandemie die nog overal aanwezig is, is er een grote stijging in het digitale gebruik. Online winkelen heeft een enorme groei gekend. Koerierbedrijven hebben overuren moeten draaien om pakjes en post te brengen bij de mensen thuis. Technologie bedrijven kennen een extra druk omdat er meer beroep wordt gedaan op IT-services. Deze digitale (r)evolutie toont ons dat beschikbaarheid van diensten zeker heel relevant is. Stel je voor dat een server van Zalando, door software problemen, uitvalt. Alle aankopen van het laatste uur zijn niet doorgekomen. Een financieel drama. De oplossing? Een standby server die inspringt in geval van downtime. Resultaat? Geen downtime, geen financieel drama, geen geknoei met corrupte data. Met dit voorbeeld wil ik op een simpele manier het belang aantonen van High Availability in clusters. Stel, er loopt iets mis, kan het probleem snel opgelost worden, zonder dat de klant of het bedrijf er iets van nadelige ervaringen aan overhoudt.

Ik wil graag Thomas Aelbrecht, mijn promotor, bedanken voor de goede begeleiding en de duidelijke feedback. Ongeveer tweewekelijks kwamen we samen om eens te overlopen hoever ik zat. Hierdoor gaf ik mijzelf telkens een deadline tegen wanneer ik bepaalde zaken verricht wou hebben.

Ook wil ik Ruben Demey, co-promotor, bedanken om bij de vragen die ik had, duidelijke antwoorden te geven waardoor ik telkens een stap dichter was bij het einde.

Tot slot wil ik ook nog mijn vriendin bedanken voor de vele steun en toeverlaat.

Veel leesplezier toegewenst!