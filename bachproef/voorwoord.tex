%%=============================================================================
%% Voorwoord
%%=============================================================================

\chapter*{\IfLanguageName{dutch}{Woord vooraf}{Preface}}
\label{ch:voorwoord}

%% TODO:
%% Het voorwoord is het enige deel van de bachelorproef waar je vanuit je
%% eigen standpunt (``ik-vorm'') mag schrijven. Je kan hier bv. motiveren
%% waarom jij het onderwerp wil bespreken.
%% Vergeet ook niet te bedanken wie je geholpen/gesteund/... heeft
Deze bachelorproef werd geschreven in het kader van het voltooien van de opleiding Toegepaste Informatica afstudeerrichting Systeem- en Netwerkbeheer. Ik heb gekozen voor dit onderwerp omdat High Availability clustering mij tijdens de opleiding altijd al interesseerde. High Availability lijkt mij iets dat steeds meer en meer relevant wordt. Zeker met de huidige Corona-crisis, is een stabiele online omgeving, cruciaal om competitief te blijven. Dit onderwerp gaf mij de kans om mijzelf 100\% te verdiepen in een onderwerp dat mij interesseerde.

Ik wil graag Thomas Aelbrecht, mijn promotor, bedanken voor de goede begeleiding en de duidelijke feedback. Ongeveer tweewekelijks kwamen we samen om eens te overlopen hoever ik zat. Hierdoor gaf ik mijzelf telkens een deadline tegen wanneer ik bepaalde zaken verricht wou hebben.

Ook wil ik Ruben Demey, co-promotor, bedanken om bij de vragen die ik had, duidelijke antwoorden te geven waardoor ik telkens een stap dichter was bij het einde.

Ik wil ook Shavawn Somers bedanken om mijn bachelorproef helemaal door te nemen en een blik te werpen op de grammatica en leesbaarheid.

Zeker wil ik mijn ouders bedanken voor de financiële en mentale steun om deze opleiding tot een goed einde te brengen.

Tot slot wil ik ook nog mijn vriendin bedanken voor de vele steun en toeverlaat.

Veel leesplezier toegewenst!