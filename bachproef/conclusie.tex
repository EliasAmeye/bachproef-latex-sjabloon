%%=============================================================================
%% Conclusie
%%=============================================================================

\chapter{Conclusie}
\label{ch:conclusie}

In dit onderzoek wordt een antwoord gegeven op de onderzoeksvraag: “Welke PostgreSQL High Availability cluster oplossing kunnen bedrijven, de dag van vandaag, gebruiken om garantie te hebben op monitoring, redundantie en failover?” Om hierop een antwoord te vinden, is een vergelijkende studie uitgevoerd tussen de verschillende oplossingen uit dit onderzoek. Hier werd dan elke oplossing afgetoetst aan de requirements.

Alle oplossingen kort omschrijven hier

Uiteindelijke oplossing omschrijven, waarom deze nu juist wel gekozen is
Had ik dit verwacht?

Wat is de meerwaarde van dit onderzoek?

Bespreken vervolgonderzoek voor toekomst over dit onderwerp.

Hierin zetten welke tool er gewonnen heeft, in dit geval Patroni.

1. Antwoord op Hoofdonderzoeksvraag

2. Antwoord op deelonderzoekvragen

% TODO: Trek een duidelijke conclusie, in de vorm van een antwoord op de
% onderzoeksvra(a)g(en). Wat was jouw bijdrage aan het onderzoeksdomein en
% hoe biedt dit meerwaarde aan het vakgebied/doelgroep? 
% Reflecteer kritisch over het resultaat. In Engelse teksten wordt deze sectie
% ``Discussion'' genoemd. Had je deze uitkomst verwacht? Zijn er zaken die nog
% niet duidelijk zijn?
% Heeft het onderzoek geleid tot nieuwe vragen die uitnodigen tot verder 
%onderzoek?


