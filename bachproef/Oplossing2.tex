%%=============================================================================
%% PostgreSQL Automatic Failover (PAF)
%%=============================================================================

\chapter{PostgreSQL Automatic Failover (PAF)}
\label{ch:PostgreSQL Automatic Failover (PAF)}

In dit hoofdstuk wordt meer informatie verleend over PostgreSQL Automatic Failover (PAF) als een PostgreSQL High Availability cluster oplossing. Er zal worden ingegaan op de verschillende requirements, waaraan voldaan moet worden. Hierbij wordt ook een score gegeven die in Hoofdstuk~\ref{ch:Verwerking resultaten} verder geanalyseerd zal worden.

%https://dzone.com/articles/managing-high-availability-in-postgresql-part-i
%In dit hoofdstuk worden nog eens voor PostgreSQL Automatic Failover (PAF) alle punten uit de functionele requirements overlopen. Hier kunnen al meer specifieke commando's en tools benoemd worden die gebruikt kunnen worden voor de opstelling van een PostgreSQL Automatic Failover (PAF) cluster.
%https://wiki.postgresql.org/images/0/07/Ha_postgres.pdf

%Inleiding

\section{\IfLanguageName{dutch}{Inleiding tot PostgreSQL Automatic Failover (PAF)}{Inleiding tot PostgreSQL Automatic Failover (PAF)}}
\label{sec:Inleiding tot PostgreSQL Automatic Failover (PAF)}

PostgreSQL Automatic Failover (PAF) is een High Availability oplossing ontwikkeld door ClusterLabs die zich bezighoudt met het beheer van High Availability in PostgreSQL clusters. Het maakt vooral gebruik van Postgres synchrone replicatie om te garanderen dat er geen gegevens verloren gaan op het moment van de failover operatie. 

PostgreSQL Automatic Failover (PAF) werkt nauw samen met de tool Pacemaker. PAF is in staat om aan Pacemaker te laten zien wat de huidige status is van een node. Bij het optreden van een storing zal Pacemaker automatisch proberen dit te herstellen.
Als de storing niet te herstellen valt, dan zal PAF zorgen voor automatische failover

Pacemaker is een service die in staat is om veel resources te beheren, en doet dit met de hulp van hun resource agents. Resource agents hebben de verantwoordelijkheid om een specifieke resource af te handelen en Pacemaker te informeren over deze resultaten.

%Bronvermelding nog bijzetten

\section{\IfLanguageName{dutch}{Requirements}{Requirements}}
\label{sec:Requirements}

\subsection{\IfLanguageName{dutch}{Must have}{Must have}}
\label{subsec:Must have}

%PostgreSQL Automatic Failover (PAF) voldoet aan de nodige requirements, zeker met de tool Pacemaker is er heel veel mogelijk qua monitoring, failover en replicatie. De score hierbij voor “Must have” is 13/15 punten.

\subsubsection{\IfLanguageName{dutch}{Ondersteuning van replicatie}{Ondersteuning van replicatie}}
\label{subsubsec:Ondersteuning van replicatie}

PostgreSQL Automatic Failover (PAF) zal niet 100\% kunnen beschermen tegen gegevensverlies. De replicatie wordt geconfigureerd met PostgreSQL. Bij asynchrone replicatie, wat de standaard is voor PostgreSQL, zullen de transacties eerst op primary node gecommitteerd worden voordat ze op de standby nodes zullen worden toegepast. In het geval van failover, zal de meest up-to-date standby node gepromoot worden door PostgreSQL Automatic Failover (PAF). Hierdoor zal gegevensverlies geminimaliseerd worden, maar niet onbestaande zijn. 

Voornamelijk zal PostgreSQL Automatic Failover (PAF) gebruik maken van synchrone replicatie om te garanderen dat er geen data verloren gaat tijdens een failover.

Hierdoor krijgt PostgreSQL Automatic Failover (PAF) voor deze requirement  een score van 4.

\subsubsection{\IfLanguageName{dutch}{Ondersteuning van failover}{Ondersteuning van failover}}
\label{subsubsec:Ondersteuning van failover}

Aan de hand van Pacemaker kan er een automatische failover plaatsvinden. PostgreSQL Automatic Failover (PAF) communiceert continu met Pacemaker over de status van de cluster en bewaakt het functioneren van de database. In geval van storing, wordt Pacemaker hierover geïnformeerd en zal het kijken om de storing op te lossen. Is de storing onoplosbaar, dan zal Pacemaker een verkiezing starten tussen de standby nodes om een opvolger van de primary node te selecteren.
Bij het configureren van de cluster wordt elke node geconfigureerd als standby node vooraleer Pacemaker bepaalt welke node de primary wordt. Dit zorgt er ook voor dat elke node weet hoe hij moet functioneren als een standby node.

“master-max” krijgt bij de configuratie van Pacemaker het aantal PostgreSQL nodes dat als primary zullen dienen op een gegeven moment. Default staat deze ingesteld op 1.

Bij “clone-max” staat de ingestelde parameter gelijk aan het aantal nodes dat PostgreSQL kunnen draaien, primary of standby. Deze parameter staat normaal gelijk aan het aantal nodes die aanwezig zijn in de cluster.

“notify=true” laat toe om te signaleren wanneer er bepaalde acties gebeuren op de node. Dit zal in contact staan met Pacemaker. Deze staat standaard op inactief, dus moet de parameter liefst expliciet nog eens vermeld worden op 'true'.

PAF maakt gebruik van ip-adres failover in plaats van het herstarten van de standby node om verbinding te maken met de nieuwe master tijdens een failover event, wat voordelig is in scenario's waar een gebruiker de standby nodes niet wil herstarten.

Door de hierboven genoemde redenen krijgt PostgreSQL Automatic Failover (PAF) een score van 4 voor deze requirement.

\subsubsection{\IfLanguageName{dutch}{Ondersteuning van monitoring}{Ondersteuning van monitoring}}
\label{subsubsec:Ondersteuning van monitoring}

Via de tool Pacemaker kan er bij PostgreSQL Automatic Failover (PAF) goed aan monitoring gedaan worden. Het zal signaleren wanneer een node niet bereikbaar is.
Het toevoegen van Watchdog als tool is hier mogelijk om ook monitoring te doen.

Hierdoor krijgt PostgreSQL Automatic Failover (PAF) voor monitoring een score van 4.

\subsection{\IfLanguageName{dutch}{Should have}{Should have}}
\label{subsec:Should have}

%Doordat er (nog) geen nieuwe release is in 2021, krijgt PostgreSQL Automatic Failover (PAF) ook 4/5 punten voor “Should have”, en geen 5/5.

\subsubsection{\IfLanguageName{dutch}{Actieve ondersteuning in 2020-2021}{Actieve ondersteuning in 2020-2021}}
\label{subsubsec:Actieve ondersteuning in 2020-2021}

De laatste release dateert van 10 maart 2020. Hieruit kan worden geconcludeerd dat ook deze oplossing nog up-to-date is en bruikbaar voor dit onderzoek. 

Hier krijgt PostgreSQL Automatic Failover (PAF) een score van 4. Een score van 5 ging behaald worden moest de laatste release voor PostgreSQL Automatic Failover (PAF) in 2021 zijn.

\subsubsection{\IfLanguageName{dutch}{Open source}{Open source}}
\label{subsubsec:Open source}

PostgreSQL Automatic Failover (PAF) is een open source High Availability oplossing voor PostgreSQL clusters.

Voor deze requirement krijgt PostgreSQL Automatic Failover (PAF) een score van 5.

\subsection{\IfLanguageName{dutch}{Could have}{Could have}}
\label{subsec:Could have}

\subsubsection{\IfLanguageName{dutch}{Grafische interface}{Grafische interface}}
\label{subsubsec:Grafische interface}

Op eerste zicht lijkt er geen algemeen gekende grafische interface aanwezig te zijn voor PostgreSQL Automatic Failover (PAF).

\subsubsection{\IfLanguageName{dutch}{Beperkte manuele interventie}{Beperkte manuele interventie}}
\label{subsubsec:Beperkte manuele interventie}

PostgreSQL Automatic Failover (PAF) kan zo geconfigureerd worden dat het weinig tot geen manuele interventie vereist. Aan de hand van bijvoorbeeld Pacemaker is dit mogelijk.

Dit was de laatste requirement waarop PostgreSQL Automatic Failover (PAF) geëvalueerd zou worden. In onderstaande tabel~\ref{table:Overzicht score PostgreSQL Automatic Failover (PAF)} is het overzicht terug te vinden van de scores per requirement. Dit zal enkel gaan over de requirements \colorbox{yellow}{“must have”} en \colorbox{orange}{“should have”}. De “could have” requirements hebben geen invloed op de eindscore.



\begin{table}[!h]
    \centering
    \begin{tabular}{ |p{6cm}||p{6cm}|  }
        \hline
        \multicolumn{2}{|c|}{Overzicht score PostgreSQL Automatic Failover (PAF)} \\
        \hline
        \colorbox{yellow}{“Must have”} & \\
        \hline
        Ondersteuning van replicatie  & 4 \\
        Ondersteuning van failover &  4 \\
        Ondersteuning van monitoring & 4 \\
        \hline
        \colorbox{orange}{“Should have”} & \\
        \hline
        Open Source &  5 \\
        Ondersteuning in 2020-2021 & 4 \\
        \hline
        \hline
        Totaal & 21/25 \\
        \hline    
    \end{tabular}
    \caption{Overzicht score PostgreSQL Automatic Failover (PAF)}
    \label{table:Overzicht score PostgreSQL Automatic Failover (PAF)}
\end{table}
