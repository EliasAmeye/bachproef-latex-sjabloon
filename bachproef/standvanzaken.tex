\chapter{\IfLanguageName{dutch}{Stand van zaken}{State of the art}}
\label{ch:stand-van-zaken}

% Tip: Begin elk hoofdstuk met een paragraaf inleiding die beschrijft hoe
% dit hoofdstuk past binnen het geheel van de bachelorproef. Geef in het
% bijzonder aan wat de link is met het vorige en volgende hoofdstuk.

% Pas na deze inleidende paragraaf komt de eerste sectiehoofding.

%Dit hoofdstuk bevat je literatuurstudie. De inhoud gaat verder op de inleiding, maar zal het onderwerp van de bachelorproef *diepgaand* uitspitten. De bedoeling is dat de lezer na lezing van dit hoofdstuk helemaal op de hoogte is van de huidige stand van zaken (state-of-the-art) in het onderzoeksdomein. Iemand die niet vertrouwd is met het onderwerp, weet nu voldoende om de rest van het verhaal te kunnen volgen, zonder dat die er nog andere informatie moet over opzoeken \autocite{Pollefliet2011}.

%Je verwijst bij elke bewering die je doet, vakterm die je introduceert, enz. naar je bronnen. In \LaTeX{} kan dat met het commando \texttt{$\backslash${textcite\{\}}} of \texttt{$\backslash${autocite\{\}}}. Als argument van het commando geef je de ``sleutel'' van een ``record'' in een bibliografische databank in het Bib\LaTeX{}-formaat (een tekstbestand). Als je expliciet naar de auteur verwijst in de zin, gebruik je \texttt{$\backslash${}textcite\{\}}.
%Soms wil je de auteur niet expliciet vernoemen, dan gebruik je \texttt{$\backslash${}autocite\{\}}. In de volgende paragraaf een voorbeeld van elk.

%\textcite{Knuth1998} schreef een van de standaardwerken over sorteer- en zoekalgoritmen. Experten zijn het erover eens dat cloud computing een interessante opportuniteit vormen, zowel voor gebruikers als voor dienstverleners op vlak van informatietechnologie~\autocite{Creeger2009}.

PostgreSQL


PostgreSQL is een open source systeem dat zich toelegt op het beheer van object-relationele databases. Het heeft meer dan 30 jaar actieve ontwikkeling en heeft een sterke reputatie op vlak van betrouwbaarheid, robuustheid van functies en prestaties (https://www.postgresql.org/). PostgreSQL biedt een uitgebreide set van functionaliteiten die een hoge mate van customisatie mogelijk maakt binnen het systeem. Dit gaat van data administratie, beveiliging, tot backup en herstel . PostgreSQL wordt regelmatig bijgewerkt door de PostgreSQL Global Development Group en bijdragers uit de community. Deze community ondersteunt zichzelf en zijn gebruikers door het aanbieden van online educatieve bronnen en communicatiekanalen, zoals daar zijn PostgreSQL wiki, online forums en officiële documentatie. Er zijn ook bedrijven die commerciële support bieden aan een prijs (https://nethosting.com/mysql-vs-postgresql-2019-showdown/ ). Volgens DB-Engines is PostgreSQL de vierde database die 
vandaag de dag het meest gebruikt wordt en de tweede meest gebruikte open source database, na MySQL (https://db-engines.com/en/ranking). DB-Engines verklaarde PostgreSQL in 2017, 2018 en 2020 het DBMS (Database management system) van het jaar (https://db-engines.com/en/system/PostgreSQL). PostgreSQL biedt veel mogelijkheden om ontwikkelaars te helpen bij het bouwen van applicaties, om beheerders te helpen bij het beschermen van data-integriteit en het bouwen van fouttolerante omgevingen, en het helpen van beheren van data, hoe groot of hoe klein de dataset ook is. PostgreSQL voldoet sinds september 2020 aan 170 van de 179 verplichte functies voor SQL:2016 Core conformiteit. Schaalbaarheid valt ook toe te schrijven aan PostgreSQL, dit zowel in de hoeveelheid data die het kan beheren, als in het aantal gelijktijdige gebruikers dat het kan accomoderen. Er zijn actieve PostgreSQL clusters in productie omgevingen die terabytes aan data beheren, en gespecialiseerde systemen die petabytes beheren (https://www.postgresql.org/about/).




High Availability


Cluster


Puppet

Puppet CTO Deepak Giridharagopal zei dat in het kielzog van de economische neergang als gevolg van de COVID-19 pandemie, meer IT-teams zwaar zullen moeten vertrouwen op automatisering. De meeste IT-teams zullen ofwel even groot blijven of worden ingekrompen. De IT-omgeving zal echter steeds complexer worden. De enige manier om IT-teams in staat te stellen meer te doen met minder is het automatiseren van meer routinetaken (https://devops.com/puppet-brings-orchestration-to-it-automation/).

Puppet is een cross-platform client-server gebaseerde toepassing die wordt gebruikt voor configuratiebeheer. Het behandelt de software en zijn configuraties op meerdere servers. Er zijn hierbij twee versies beschikbaar. De ene is open-source, de andere is een betalende, commerciële versie. Het werkt op zowel Linux als op Windows. Het gebruikt een declaratieve aanpak om updates, installaties en andere taken te automatiseren. De software kan systemen configureren met behulp van bestanden die manifesten worden genoemd. Een manifest bevat instructies voor een groep of type server(s) die wordt/worden beheerd(https://www.liquidweb.com/kb/what-is-puppet-and-what-role-does-it-play-in-devops/). 

Wat is configuratiebeheer nu juist? Configuratiebeheer onderhoudt en bepaalt productkenmerken door fysieke en functionele attributen, ontwerp, vereisten en operationele informatie op te slaan gedurende de levenscyclus van een server. 

Puppet maakt gebruik van de beschrijvende programmeertaal Ruby. Ruby is een dynamische, open source programmeertaal met de nadruk op eenvoud en productiviteit (https://www.ruby-lang.org/en/).

Vroeger werden software en systemen door systeembeheerders manueel opgezet en geconfigureerd. Maar toen het te beheren aantal servers snel toenam, moest er gezocht worden naar een manier om die processen te automatiseren, om dan zo tijd te besparen en de nauwkeurigheid te vergroten. Puppet is uit deze zoektocht ontstaan.

Puppet werkt aan de hand van een eenvoudig client/server architectuur workflow proces. Hierin bestaat er een master server die alle informatie bevat over de configuraties van de verschillende nodes aanwezig. Het slaat deze configuraties op in manifestbestanden op een een centrale server, genaamd de Puppet master, en voert deze manifesten uit op de remote client servers genaamd agents (https://www.liquidweb.com/kb/what-is-puppet-and-what-role-does-it-play-in-devops/).

