%%=============================================================================
%% Conclusie
%%=============================================================================

\chapter{Proof of Concept}
\label{ch:Proof of Concept}

In dit hoofdstuk wordt de opbouw van de Patroni cluster uitgelegd. Hierin zal worden gekeken hoe de cluster aan de functionele requirements, namelijk redundantie/replicatie, failover en monitoring voldoet.

\section{\IfLanguageName{dutch}{Omgeving}{Environment}}
\label{sec:Omgeving}
De opzet van deze proof of concept zal gebeuren in een lokale Linux-omgeving, namelijk Ubuntu 21.04.
De cluster zal bestaan uit drie virtuele machines die zullen draaien op Ubuntu Xenial 16.04. De eerste virtuele machine (pgServer) is een servernode waarop Consul zal draaien, samen met pgBouncer en een HAProxy. De andere twee virtuele machines (pgNode1 en pgNode2) zullen PostgreSQL, Patroni en een Consul Agent draaien.

De cluster zal dus zo geconfigueerd worden dat er één primary node is met een standby node die asynchrone streaming replicatie verricht.

\section{\IfLanguageName{dutch}{Prerequisites}{Prerequisites}}
\label{sec:Prerequisites}

\subsection{\IfLanguageName{dutch}{Vagrant}{Vagrant}}
\label{subsec:Vagrant}
Bij de opzet van cluster zal gebruik worden gemaakt van Vagrant. Vagrant is een tool voor het bouwen en beheren van virtuele machine-omgevingen~\autocite{Kalow2020}.

\definecolor{dkgreen}{rgb}{0,0.6,0}
\definecolor{gray}{rgb}{0.5,0.5,0.5}
\definecolor{mauve}{rgb}{0.58,0,0.82}

\lstset{frame=tb,
    language=Java,
    aboveskip=3mm,
    belowskip=3mm,
    showstringspaces=false,
    columns=flexible,
    basicstyle={\small\ttfamily},
    numbers=none,
    numberstyle=\tiny\color{gray},
    keywordstyle=\color{blue},
    commentstyle=\color{dkgreen},
    stringstyle=\color{mauve},
    breaklines=true,
    breakatwhitespace=true,
    tabsize=3
}

In onderstaande code snippet is de code voor de Vagrantfile zichtbaar. Hierin wordt bepaald welke servers aangemaakt worden en welke opties zij bezitten. Zo zal elke node een RAM-geheugen hebben van 1 GB (1024 MB). Ook het ip-adres zal hierin worden toegewezen per node. Er wordt ook een file meegegeven die zal worden uitgevoerd in de server. Deze file bevat de opbouw van de cluster, zoals het installeren van Consul, het configureren van Patroni op de nodes, en de setup van HAProxy en pgBouncer.

\begin{lstlisting}
# -*- mode: ruby -*-
# vi: set ft=ruby :

VAGRANTFILE_API_VERSION = "2"

Vagrant.configure(VAGRANTFILE_API_VERSION) do |config|
    config.vm.box = "ubuntu/xenial64"
    config.vm.provider "virtualbox" do |v|
        v.customize ["modifyvm", :id, "--memory", 1024]
        v.gui = true
    end

    config.vm.define :pgServer do |pgServer_config|
        pgServer_config.vm.hostname = 'pgServer'
        pgServer_config.vm.network :private_network, ip: "172.16.0.11"
        pgServer_config.vm.provision :shell, :path => "postgresql-cluster-setup.sh"
    end
    config.vm.define :pgNode1, primary: true do |pgNode1_config|
        pgNode1_config.vm.hostname = 'pgNode1'
        pgNode1_config.vm.network :private_network, ip: "172.16.0.22"
        pgNode1_config.vm.provision :shell, :path => "postgresql-cluster-setup.sh"
    end
    config.vm.define :pgNode2 do |pgNode2_config|
        pgNode2_config.vm.hostname = 'pgNode2'
        pgNode2_config.vm.network :private_network, ip: "172.16.0.33"
        pgNode2_config.vm.provision :shell, :path => "postgresql-cluster-setup.sh"
    end
end
\end{lstlisting}


\subsection{\IfLanguageName{dutch}{VirtualBox}{VirtualBox}}
\label{subsec:VirtualBox}
In deze proof of concept zal gebruik gemaakt worden van virtuele machines via VirtualBox.

\subsection{\IfLanguageName{dutch}{Python3-pip}{Python3-pip}}
\label{subsec:Python3-pip}

% niet meer recent tegengekomen


\section{\IfLanguageName{dutch}{pgServer}{pgServer}}
\label{sec:pgServer}
Tijdens de configuratie van de cluster zal pgServer '172.16.0.11' als ip adres hebben.

Als eerste stap moeten de juiste packages geïnstalleerd worden op pgServer. Hier gaat het over Consul, pgBouncer en HAProxy.

Hierna zullen we voor Consul een service toevoegen die bij het opstarten van de server wordt uitgevoerd. Belangrijk is dat bij het aanmaken van deze service, de service gestart en ge-enabled moet zijn. In deze service verwijzen we naar de config file voor de Consul server. Deze ziet er als volgt uit.

\begin{lstlisting}
    {
        "advertise_addr": "172.16.0.11",
        "bind_addr": "172.16.0.11",
        "bootstrap": false,
        "bootstrap_expect": 1,
        "server": true,
        "client_addr": "0.0.0.0",
        "node_name": "pgServer",
        "datacenter": "dc1",
        "data_dir": "/var/consul/server",
        "domain": "consul",
        "encrypt": "/q/vkVS+My2nl8Zk/8csuQ==",
        "log_level": "INFO",
        "enable_syslog": true,
        "rejoin_after_leave": true,
        "ui_dir": "/var/consul/ui",
        "leave_on_terminate": false,
        "skip_leave_on_interrupt": false
    }
    
 \end{lstlisting}

Bij het installeren van de HAProxy wordt doorverwezen naar een configuratiefile waarin de verschillende opties worden meegegeven.

\begin{lstlisting}
global
    maxconn 100

defaults
    log global
    mode tcp
    retries 2
    timeout client 30m
    timeout connect 4s
    timeout server 30m
    timeout check 5s

listen stats
    mode http
    bind *:7000
    stats enable
    stats uri /

listen postgres
    bind *:5000
    option httpchk
    http-check expect status 200
    default-server inter 3s fall 3 rise 2 on-marked-down shutdown-sessions
    server postgresql_pgNode1_172.0.16.22 172.16.0.22:5432 maxconn 100 check port 8008
    server postgresql_pgNode2_172.0.16.33 172.16.0.33:5432 maxconn 100 check port 8008
\end{lstlisting}


\section{\IfLanguageName{dutch}{pgNode1}{pgNode1}}
\label{sec:pgNode1}
Tijdens de configuratie van de cluster zal pgNode1 '172.16.0.22' als ip adres hebben.


Idem als bij pgServer is de eerste stap het installeren van de juiste packages. Hier gaat het over Consul Agent, Patroni en PostgreSQL uiteraard.

Dan

Hierna zal opnieuw voor Consul een service toegevoegd worden die bij het opstarten van de server wordt uitgevoerd. Belangrijk is dat bij het aanmaken van deze service, de service gestart en ge-enabled moet zijn. In deze service verwijzen we naar de config file voor de Consul agent (cliënt). Deze ziet er als volgt uit.

\begin{lstlisting}
    {
        "server": false,
        "datacenter": "dc1",
        "data_dir": "/var/consul/client",
        "ui_dir": "/var/consul/ui",
        "encrypt": "/q/vkVS+My2nl8Zk/8csuQ==", 
        "log_level": "INFO",
        "enable_syslog": true,
        "start_join": ["172.16.0.11"],
        "bind_addr": "172.16.0.22"
    }  
\end{lstlisting}


\section{\IfLanguageName{dutch}{pgNode2}{pgNode2}}
\label{sec:pgNode2}
Tijdens de configuratie van de cluster zal pgNode2 '172.16.0.33' als ip adres hebben.

Idem als bij pgServer is de eerste stap het installeren van de juiste packages. Hier gaat het over Consul Agent, Patroni en PostgreSQL uiteraard.

Dan 

Hierna zal opnieuw voor Consul een service toegevoegd worden die bij het opstarten van de server wordt uitgevoerd. Belangrijk is dat bij het aanmaken van deze service, de service gestart en ge-enabled moet zijn. In deze service verwijzen we naar de config file voor de Consul agent (cliënt). Deze ziet er als volgt uit.

\begin{lstlisting}
    {
        "server": false,
        "datacenter": "dc1",
        "data_dir": "/var/consul/client",
        "ui_dir": "/var/consul/ui",
        "encrypt": "/q/vkVS+My2nl8Zk/8csuQ==", 
        "log_level": "INFO",
        "enable_syslog": true,
        "start_join": ["172.16.0.11"],
        "bind_addr": "172.16.0.33"
    }
\end{lstlisting}


\begin{tabular}{ |p{6cm}||p{6cm}|  }
    \hline
    \multicolumn{2}{|c|}{Ip-adressen} \\
    \hline
    pgServer & 172.16.0.11 \\
    \hline
    pgNode1 & 172.16.0.22 \\
    \hline
    pgNode2 & 172.16.0.33 \\
    \hline
\end{tabular}



\caption[Tabel 6: Ip-adressen Cluster]



\section{\IfLanguageName{dutch}{Testing functionele requirements}{Testing functionele requirements}}
\label{sec:Testing functionele requirements}

3. PostgreSQL setup file doorlopen om opzet van POC te tonen

4. Eventueel ook services tonen

5. Consul config tonen

6. Hoe failover testen

7. Monitoring tonen

10. Replicatie tonen


Stappen doorlopen met code snippets

Via welke manier ik het ga opzetten en aan welke testen ik wil voldoen.

1a. 2 nodes die we pg01 en pg02 gaan noemen, ip adressen toewijzen

Deze hebben beide postgres, patroni, consul
ip adres toewijzen

1b. 3de node, pg03, voor Consul, HAProxy, pgBouncer


2. Opzetten Yaml files voor configuratie Patroni


(Puppet kijken naar tijd)
