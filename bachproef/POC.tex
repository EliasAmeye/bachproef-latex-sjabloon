%%=============================================================================
%% Conclusie
%%=============================================================================

\chapter{Proof of Concept}

\label{ch:Proof of Concept}



Opzet Cluster Patroni

Adhv Vagrant met virtualbox

Stappen doorlopen met code snippets

Via welke manier ik het ga opzetten en aan welke testen ik wil voldoen.

In welke omgeving: Linux Ubuntu

Pre-requisites:

vagrant
python3-pip
virtualbox


1. Opzetten Vagrant file

1a. 2 nodes die we pg01 en pg02 gaan noemen, ip adressen toewijzen

Deze hebben beide postgres, patroni, consul
ip adres toewijzen

1b. 3de node, pg03, voor Consul, HAProxy, pgBouncer


2. Opzetten Yaml files voor configuratie Patroni

scope: clustertest
namespace: /nsclustertest
name: pg01

log:
level: INFO
dir: /var/log/patroni
file_size: 10485760 # 10MB

restapi:
listen: 172.28.33.11:8008
connect_address: 172.28.33.11:8008

consul:
host: 172.28.33.13:8500

bootstrap:
dcs:
ttl: 30
loop_wait: 10
retry_timeout: 10
maximum_lag_on_failover: 1048576 # 1MB
master_start_timeout: 300
postgresql:
use_pg_rewind: true
parameters:
archive_command: 'exit 0'
archive_mode: 'on'
autovacuum: 'on'
checkpoint_completion_target: 0.6
#checkpoint_segments: 10
checkpoint_warning: 300
#data_directory: '/var/lib/postgresql/patroni'
datestyle: 'iso, mdy'
default_text_search_config: 'pg_catalog.english'
effective_cache_size: '128MB'
#external_pid_file: '/var/run/postgresql/postgresql-main.pid'
#hba_file: '/var/lib/postgresql/patroni/pg_hba.conf'
hot_standby: 'on'
#ident_file: '/var/lib/postgresql/patroni/pg_ident.conf'
include_if_exists: 'repmgr_lib.conf'
lc_messages: 'C'
listen_addresses: '*'
log_autovacuum_min_duration: 0
log_checkpoints: 'on'
logging_collector: 'on'
log_min_messages: INFO
log_filename: 'postgresql.log'
log_connections: 'on'
log_directory: '/var/log/postgresql'
log_disconnections: 'on'
log_line_prefix: '%t [%p]: [%l-1] user=%u,db=%d,app=%a '
log_lock_waits: 'on'
log_min_duration_statement: 0
log_temp_files: 0
maintenance_work_mem: '128MB'
max_connections: 101
max_wal_senders: 5
port: 5432
shared_buffers: '128MB'
shared_preload_libraries: 'pg_stat_statements'
#ssl: on
#ssl_cert_file: '/etc/ssl/certs/ssl-cert-snakeoil.pem'
#ssl_key_file: '/etc/ssl/private/ssl-cert-snakeoil.key'
unix_socket_directories: '/var/run/postgresql'
wal_buffers: '8MB'
wal_keep_segments: '200'
wal_level: 'replica'
work_mem: '128MB'

initdb:
- encoding: UTF8
- data-checksums

pg_hba:
- host replication replicator 127.0.0.1/32 md5
- host replication replicator 172.28.33.11/0 md5
- host replication replicator 172.28.33.12/0 md5
- host all postgres 172.28.33.11/32 trust
- host all postgres 172.28.33.12/32 trust
- host all postgres 172.28.33.13/32 trust
- host all all 0.0.0.0/0 md5
- local all all peer

users:
admin:
password: admin
options:
- createrole
- createdb

postgresql:
listen: 127.0.0.1,172.28.33.11:5432
connect_address: 172.28.33.11:5432
data_dir: /var/lib/postgresql/patroni
pgpass: /tmp/pgpass
#    create_replica_methods: [] # supports: pg_basebackup, WAL-E, pgBackRest, Barman
use_unix_socket: true
authentication:
replication:
username: replicator
password: rep-pass
superuser:
username: postgres
password: secretpassword
#    custom_conf: '/etc/postgresql/9.6/main/postgresql.custom.conf'
parameters:
unix_socket_directories: '/var/run/postgresql'
wal_compression: on

tags:
nofailover: false
noloadbalance: false
clonefrom: false
nosync: false

watchdog:
mode: off



Dan in bijlage waarschijnlijk elk bestand zetten met code in?
of linken naar 




(Puppet kijken naar tijd)
