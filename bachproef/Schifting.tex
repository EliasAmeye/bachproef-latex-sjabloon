%%=============================================================================
%% Schifting 
%%=============================================================================

\chapter{\IfLanguageName{dutch}{Schifting}{Schifting}}
\label{ch:schifting}


\section{\IfLanguageName{dutch}{Requirements}{Requirements}}
\label{sec:Requirements}
Om te kunnen bepalen welke requirements er nodig zijn,

Elk van deze High Available PostgreSQL cluster oplossingen zal worden afgetoetst aan de requirements om ze op deze manier te evalueren. In samenspraak met Ruben Demey zijn er drie functionele requirements. Bij elke oplossing worden de requirements afgetoetst en krijgen ze een score van 1 (slecht) tot 5 (uitstekend).
% Aan de hand van MosCOW mss wel


\subsection{\IfLanguageName{dutch}{Functionele Requirements}{Functionele Requirements}}
\label{subsec:Functionele Requirements}
Een functionele requirement beschrijft hoe het systeem moet werken en wat het moet kunnen.

\subsubsection{\IfLanguageName{dutch}{Redundancy}{Redundancy}}
\label{subsubsec:Redundancy}
Om High Availability te bereiken moeten clusters aan bepaalde eisen voldoen. Het moet over redundantie beschikken om single points of failure te vermijden. 


\subsubsection{\IfLanguageName{dutch}{Failover}{Failover}}
\label{subsubsec:Failover}
Het opzetten van failover (replicatie) biedt de nodige redundantie om High Availability mogelijk te maken door ervoor te zorgen dat standby nodes beschikbaar zijn als de master of primary node ooit uitvalt. 

\subsubsection{\IfLanguageName{dutch}{Monitoring}{Monitoring}}
\label{subsubsec:Monitoring}
Monitoring is belangrijk omdat het op deze manier actief storingen kan opmerken en opsporen. Monitoring checkt de algemene gezondheid en beschikbaarheid van een systeem.


\subsection{\IfLanguageName{dutch}{Niet-functionele Requirements}{Niet-functionele Requirements}}
\label{subsec:Niet-functionele Requirements}
Een niet-functionele requirement is een kwaliteitseis voor het systeem. Dit gaat dan meer over voorkeuren.

\subsubsection{\IfLanguageName{dutch}{Open source}{Open source}}
\label{subsubsec:Open source}
Open source verwijst naar source code die publiekelijk toegankelijk is en die door iedereen gebruikt mag worden zonder nodige licentie~\autocite{2021c}. %(https://opensource.com/resources/what-open-source)

\subsubsection{\IfLanguageName{dutch}{Anno 2020-2021}{Anno 2020-2021}}
\label{subsubsec:Anno 2020-2021}
Een belangrijke requirement is dat de oplossing up-to-date moet zijn. In dit onderzoek zal er geen gebruik gemaakt worden van achterhaalde, oude tools. Het is belangrijk dat de tool futureproof kan zijn en zeker nog jaren kan meegaan.

\subsubsection{\IfLanguageName{dutch}{Grafische interface}{Grafische interface}}
\label{subsubsec:Grafische interface}
Een gemakkelijk te gebruiken interface is zeker een meerwaarde bij het monitoren en beheren van een cluster. Als er geen specialist aanwezig is, kan de grafische interface soms voor wat extra duidelijkheid zorgen.


\section{\IfLanguageName{dutch}{Indelen requirements volgens MoSCoW-techniek}{Indelen requirements volgens MoSCoW-techniek}}
\label{sec:Indelen requirements volgens MoSCoW-techniek}

Na het opstellen van de verschillende requirements kunnen we deze nog eens indelen volgens de MoSCoW-techniek~\autocite{Ahmad2017}. Hierin worden de requirements geprioritiseerd. Hierin zijn de 'Must have'-requirements verplicht aanwezig. De 'Should have'-requirements zijn geen verplichting, maar zijn het liefst wel aanwezig in de keuze van een oplossing. De 'Could have'-requirements zijn volledig optioneel en zijn dus niet geheel relevant bij het kiezen van de oplossing. Het 'Wont have' aspect laten we hier achterwege, omdat dit geen invloed zal hebben op de keuze van een oplossing.

\begin{description}%[font=$\bullet$\scshape\bfseries]
    \item[$\bullet$ Must have] 
        \begin{description}
            \item
            \item[$\cdot$] Ondersteuning van redundancy
            \item[$\cdot$] Ondersteuning van failover
            \item[$\cdot$] Ondersteuning van monitoring
        \end{description}
    \item[$\bullet$ Should have]
            \begin{description}
        \item
        \item[$\cdot$] Ondersteuning in 2020-2021
        \item[$\cdot$] Open source
    \end{description}
    \item[$\bullet$ Could have]
                \begin{description}
        \item
        \item[$\cdot$] Eenvoudig te gebruiken interface

    \end{description}
\end{description}

\section{\IfLanguageName{dutch}{Hoe beantwoorden de verschillende oplossingen aan de bovenstaande requirements}{Hoe beantwoorden de verschillende oplossingen aan de requirements}}
\label{sec:Hoe beantwoorden de verschillende oplossingen aan de bovenstaande requirements}

In dit gedeelte worden de verschillende oplossingen vergeleken aan de hand van de vooropgestelde requirements. Wanneer een oplossing een zeer brede implementatie heeft van een requirement zal deze aan de hand van een puntensysteem een 5/5 krijgen. Wanneer er niet voldoende implementatie aanwezig is voor deze oplossing, zal dit een 1/5 zijn. De functionele requirements zullen zwaarder doorwegen dan de niet-functionele requirements. Hierbij zal het totaal aantal punten op 20 staan, waarbij de functionele requirements 75\% van de totaalscore bevatten. Dit staat gelijk aan 15/20. De niet-functionele requirements staan dus maar op 25\%, wat gelijk staat aan 5/20 van de totaalscore. Hierbij kan er een goed beeld gevormd worden welke oplossing er functioneler is. Belangrijke niet-functionele requirement is dat de oplossing up-to-date moet zijn, anno 2020-2021. Als de oplossing hier niet aan voldoet, dan zal deze niet doorkomen, of gebruikt worden als proof of concept.

% https://scalegrid.io/blog/managing-high-availability-in-postgresql-part-3/#:~:text=Patroni%20ensures%20the%20end%2Dto,be%20customized%20to%20your%20needs.

% https://www.slideshare.net/ScaleGrid/whats-the-best-postgresql-high-availability-framework-paf-vs-repmgr-vs-patroni-infographic

%In deze links worden 3 oplossingen vergeleken met elkaar




\subsection{\IfLanguageName{dutch}{Oplossing 1: Patroni}{Oplossing 1: Patroni}}
\label{subsec: Oplossing 1: Patroni}


\subsubsection{\IfLanguageName{dutch}{Functionele Requirements}{Functionele Requirements}}
\label{subsubsec:Functionele Requirements}

Voor replicatie opties bestaan er verschillende tools zoals barman, Wal-E, die zullen helpen om nodes toe te voegen aan de cluster. Er kan ook gekozen worden van waar sommige nodes hun data zullen halen.

Patroni heeft een automatische failover-functie. Hierbij wordt automatisch gekeken welke node de te vervangen node zal vervangen. Na failover zal ook automatisch de gefaalde node terug in de cluster komen.

Patroni heeft ook een tool genaamd, ETCD waarmee de gezondheid van een PostgreSQL cluster, status van knooppunten en andere informatie van de cluster wordt bijgehouden. Tools die hiervoor ook gebruikt worden zijn Consul en Zookeeper.

Er zijn verschillende soorten tools beschikbaar voor de drie opgestelde functionele requirements. Aan de hand hiervan krijgt Patroni een 12/15.

\subsubsection{\IfLanguageName{dutch}{Niet-functionele Requirements}{Niet-functionele Requirements}}
\label{subsubsec:Niet-functionele Requirements}
Patroni is ontwikkeld door Zalando en is volledig open source. De volledige source code is online te vinden op github.com en wordt nog steeds geupdate. Dit is ook een belangrijke vereiste waaraan de oplossing moet doen. Er worden  nog steeds releases gedaan. De laatste release was, tijdens dit schrijven, op 22 februari 2021.

Aangezien Patroni volledig open source is en het een oplossing is die volledig up-to-date gehouden wordt, krijgt Patroni voor niet-functionele requirements een 5/5.



\subsection{\IfLanguageName{dutch}{Oplossing 2: Pgpool-II}{Oplossing 2: Pgpool-II}}
\label{subsec: Oplossing 2: Pgpool-II}

\subsubsection{\IfLanguageName{dutch}{Functionele Requirements}{Functionele Requirements}}
\label{subsubsec:Functionele Requirements}

Pgpool-II beschikt over een tool, genaamd Watchdog, dat een subproces is van Pgpool-II dat dient om High Availability toe te voegen. Watchdog wordt gebruikt om single point of failure op te lossen door meerdere Pgpool-II nodes te coördineren. Het coördineert meerdere Pgpool-II nodes door informatie met elkaar uit te wisselen.

Watchdog kan ook remote de gezondheid controleren van de node waarop het is geïnstalleerd door de verbinding met upstream nodes te monitoren. Als de monitoring faalt, behandelt watchdog dit als het falen van een lokale Pgpool-II node en zal het de nodige acties ondernemen.

Pgpool-II kan meerdere nodes beheren en hierin replicatie opzetten om High Availability te verkrijgen.

Pgpool-II voorziet ook automatische failover.

Pgpool-II heeft aan de hand van Watchdog en Replication manieren om aan de vooropgesteld functionele requirements te voldoen. Het aantal beschkbare tools lijken wel wat minder aanwezig te zijn. Pgpool-II krijgt hierdoor een 10/15.

\subsubsection{\IfLanguageName{dutch}{Niet-functionele Requirements}{Niet-functionele Requirements}}
\label{subsubsec:Niet-functionele Requirements}

Pgpool-II is een open source project. De source code wordt onderhouden door een git-repository.
De laatste release van pgpool-II was op 26 november 2020. Hieruit kunnen we ook zeggen dat Pgpool-II nog een geldige, bruikbare oplossing.
Aan de hand hiervan krijgt Pgpool-II een 4/5 voor niet-functionele requirements.

\subsection{\IfLanguageName{dutch}{Oplossing 3: PostgreSQL Automatic Failover (PAF)}{Oplossing 3: PostgreSQL Automatic Failover (PAF)}}
\label{subsec:Oplossing 3: PostgreSQL Automatic Failover (PAF)}

\subsubsection{\IfLanguageName{dutch}{Functionele Requirements}{Functionele Requirements}}
\label{subsubsec:Functionele Requirements}

PostgreSQL Automatic Failover (PAF) werkt nauw samen met de tool Pacemaker. PAF is in staat om aan Pacemaker te laten zien wat de huidige status is van een node. Bij het optreden van een storing zal Pacemaker automatisch proberen dit te herstellen.
Als de storing niet te herstellen valt, dan zal PAF zorgen voor automatische failover.

PostgreSQL Automatic Failover (PAF) maakt gebruik van synchrone replicatie om te garanderen dat er geen data verloren gaat tijdens een failover.

PostgreSQL Automatic Failover (PAF) voldoet aan de nodige requirements, zeker met de tool Pacemaker is er heel veel mmogelijk. De score toegekend is 12/15

\subsubsection{\IfLanguageName{dutch}{Niet-functionele Requirements}{Niet-functionele Requirements}}
\label{subsubsec:Niet-functionele Requirements}

Ook PostgreSQL Automatic Failover (PAF) is open source.
De laatste release was op 10 maart 2020. Hieruit kunnen we concluderen dat ook deze oplossing nog up-to-date is en bruikbaar voor dit onderzoek.
Op basis hiervan heeft PostgreSQL Automatic Failover (PAF) ook een 4/5 voor niet-functionele requirements.


\subsection{\IfLanguageName{dutch}{Oplossing 4: Replication Manager (RepMgr)}{Oplossing 4: Replication Manager (RepMgr)}}
\label{subsec:Oplossing 4: Replication Manager (RepMgr)}

\subsubsection{\IfLanguageName{dutch}{Functionele Requirements}{Functionele Requirements}}
\label{subsubsec:Functionele Requirements}
Replication Manager (RepMgr) is een oplossing ontwikkeld voor het beheren van replicatie en failover van PostgreSQL clusters. Het biedt de tools aan om replicatie van PostgreSQL op te zetten, te configureren, te beheren en te monitoren. Het laat ook toe om handmatige omschakeling en failover taken uit te voeren met behulp van repmgr utility. Dit is een gratis tool die ondersteuning en verbetering biedt van PostgreSQL's ingebouwde streaming replicatie.
Voorbeelden van tools zijn repmgr en repmgrd.
Replication Manager (RepMgr) biedt ook de tools om primary en standby nodes op te zetten, en om in geval van faalscenario automatische failover te doen.

de functionaliteiten van Replication Manager (RepMgr) sluiten voldoende aan op de vooropgestelde functionele requirements en krijgt hiervoor een 12/15.

\subsubsection{\IfLanguageName{dutch}{Niet-functionele Requirements}{Niet-functionele Requirements}}
\label{subsubsec:Niet-functionele Requirements}
Replication Manager (RepMgr) is open source, ontwikkeld door 2ndQuadrant.
De laatste release was op 22 oktober 2020.

Replication Manager (RepMgr) krijgt hierdoor een score van 4/5 voor niet-functionele requirments.

%4/5 want laatste release in 2020


\subsection{\IfLanguageName{dutch}{Oplossing 5}{Oplossing 5}}
\label{subsec:Oplossing 5}

\subsection{\IfLanguageName{dutch}{Oplossing 6}{Oplossing 6}}
\label{subsec:Oplossing 6}


\section{\IfLanguageName{dutch}{Resultatenanalyse}{Resultatenanalyse}}
\label{sec:Resultatenanalyse}

% Hierin zeggen hoeveel punten elke oplossing heeft enz