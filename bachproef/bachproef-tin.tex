%===============================================================================
% LaTeX sjabloon voor de bachelorproef toegepaste informatica aan HOGENT
% Meer info op https://github.com/HoGentTIN/bachproef-latex-sjabloon
%===============================================================================

\documentclass{bachproef-tin}

\usepackage{hogent-thesis-titlepage} % Titelpagina conform aan HOGENT huisstijl

%%---------- Documenteigenschappen ---------------------------------------------
% TODO: Vul dit aan met je eigen info:

% De titel van het rapport/bachelorproef
\title{High Availability oplossingen voor PostgreSQL: een vergelijkende studie en proof of concept}

% Je eigen naam
\author{Elias Ameye}

% De naam van je promotor (lector van de opleiding)
\promotor{Thomas Aelbrecht}

% De naam van je co-promotor. Als je promotor ook je opdrachtgever is en je
% dus ook inhoudelijk begeleidt (en enkel dan!), mag je dit leeg laten.
\copromotor{Ruben Demey}

% Indien je bachelorproef in opdracht van/in samenwerking met een bedrijf of
% externe organisatie geschreven is, geef je hier de naam. Zoniet laat je dit
% zoals het is.
\instelling{---}

% Academiejaar
\academiejaar{2020-2021}

% Examenperiode
%  - 1e semester = 1e examenperiode => 1
%  - 2e semester = 2e examenperiode => 2
%  - tweede zit  = 3e examenperiode => 3
\examenperiode{2}

%===============================================================================
% Inhoud document
%===============================================================================

\begin{document}

%---------- Taalselectie -------------------------------------------------------
% Als je je bachelorproef in het Engels schrijft, haal dan onderstaande regel
% uit commentaar. Let op: de tekst op de voorkaft blijft in het Nederlands, en
% dat is ook de bedoeling!

%\selectlanguage{english}

%---------- Titelblad ----------------------------------------------------------
\inserttitlepage

%---------- Samenvatting, voorwoord --------------------------------------------
\usechapterimagefalse
%%=============================================================================
%% Voorwoord
%%=============================================================================

\chapter*{\IfLanguageName{dutch}{Woord vooraf}{Preface}}
\label{ch:voorwoord}

%% TODO:
%% Het voorwoord is het enige deel van de bachelorproef waar je vanuit je
%% eigen standpunt (``ik-vorm'') mag schrijven. Je kan hier bv. motiveren
%% waarom jij het onderwerp wil bespreken.
%% Vergeet ook niet te bedanken wie je geholpen/gesteund/... heeft

In dit onderzoek zal ik het hebben over High Availability oplossingen en eht belang ervan in PostgreSQL clusters. Met de Coronapandemie die nog overal aanwezig is, is er een grote stijging in het digitale gebruik. Online winkelen heeft een enorme groei gekend. Koerierbedrijven hebben overuren moeten draaien om pakjes en post te brengen bij de mensen thuis. Technologie bedrijven kennen een extra druk omdat er meer beroep wordt gedaan op IT-services. Deze digitale (r)evolutie toont ons dat beschikbaarheid van diensten zeker heel relevant is. Stel je voor dat een server van Zalando, door software problemen, uitvalt. Alle aankopen van het laatste uur zijn niet doorgekomen. Een financieel drama. De oplossing? Een standby server die inspringt in geval van downtime. Resultaat? Geen downtime, geen financieel drama, geen geknoei met corrupte data. Met dit voorbeeld wil ik het belang aantonen van High Availability in clusters. Stel, er loopt iets mis, kan het probleem snel opgelost worden, zonder dat de klant of het bedrijf er iets van nadelige ervaringen aan overhoudt.

Ik wil graag Thomas Aelbrecht, mijn promotor, bedanken voor de goede begeleiding en de duidelijke feedback. Ongeveer tweewekelijks kwamen we samen om eens te overlopen hoever ik zat. Hierdoor gaf ik mijzelf telkens een deadline tegen wanneer ik bepaalde zaken verricht wou hebben.

Ook wil ik Ruben Demey, co-promotor, bedanken om bij de vragen die ik had, duidelijke antwoorden te geven waardoor ik telkens een stap dichter was bij het einde.

Tot slot wil ik ook nog mijn vriendin bedanken voor de steun en toeverlaat.
%%=============================================================================
%% Samenvatting
%%=============================================================================

% TODO: De "abstract" of samenvatting is een kernachtige (~ 1 blz. voor een
% thesis) synthese van het document.
%
% Deze aspecten moeten zeker aan bod komen:
% - Context: waarom is dit werk belangrijk?
% - Nood: waarom moest dit onderzocht worden?
% - Taak: wat heb je precies gedaan?
% - Object: wat staat in dit document geschreven?
% - Resultaat: wat was het resultaat?
% - Conclusie: wat is/zijn de belangrijkste conclusie(s)?
% - Perspectief: blijven er nog vragen open die in de toekomst nog kunnen
%    onderzocht worden? Wat is een mogelijk vervolg voor jouw onderzoek?
%
% LET OP! Een samenvatting is GEEN voorwoord!

%%---------- Nederlandse samenvatting -----------------------------------------
%
% TODO: Als je je bachelorproef in het Engels schrijft, moet je eerst een
% Nederlandse samenvatting invoegen. Haal daarvoor onderstaande code uit
% commentaar.
% Wie zijn bachelorproef in het Nederlands schrijft, kan dit negeren, de inhoud
% wordt niet in het document ingevoegd.

\IfLanguageName{english}{%
\selectlanguage{dutch}
\chapter*{Samenvatting}
\lipsum[1-4]
\selectlanguage{english}
}{}

%%---------- Samenvatting -----------------------------------------------------
% De samenvatting in de hoofdtaal van het document

\chapter*{\IfLanguageName{dutch}{Samenvatting}{Abstract}}

Dit onderzoek kan dienen als basis voor de keuze tussen verschillende PostgreSQL High Availability cluster oplossingen. Doordat PostgreSQL een zeer populair open source object-relationele databank systeem is, en het geen tekenen geeft van in gebruikers te minderen, is het zeker interessant om te kijken hoe een bedrijf in 2020-2021 zijn High Availability clusters kan updaten, of creëren. In dit onderzoek ligt de focus vooral op High Availability in een PostgreSQL omgeving. Er wordt gekeken naar verschillende oplossingen zoals Patroni, PostgreSQL Automatic Failover (PAF), Pgpool-II en Replication Manager (repmgr). Deze oplossingen worden aan de hand van verschillende requirements met elkaar vergeleken. Hieruit brengt dit onderzoek één oplossing voor die gebruikt kan worden om High Availability te implementeren. In dit onderzoek vind u een inleiding tot High Availability in een PostgreSQL cluster met hieraan verbonden de huidige stand van zaken. Hierna worden vier oplossingen met elkaar vergeleken aan de hand van vooropgestelde requirements. Aan de hand van deze requirements krijgt elke oplossing een score toegewezen waarop bepaald zal worden met welke oplossing er verder gewerkt zal worden. De ultieme oplossing uit dit onderzoek blijkt Patroni te zijn die zeer goed aan de verschillende requirements voldoet. Tweede keus komt te liggen bij PostgreSQL Automatic Failover (PAF).

Nog af te werken.

%---------- Inhoudstafel -------------------------------------------------------
\pagestyle{empty} % Geen hoofding
\tableofcontents  % Voeg de inhoudstafel toe
\cleardoublepage  % Zorg dat volgende hoofstuk op een oneven pagina begint
\pagestyle{fancy} % Zet hoofding opnieuw aan

%---------- Lijst figuren, afkortingen, ... ------------------------------------

% Indien gewenst kan je hier een lijst van figuren/tabellen opgeven. Geef in
% dat geval je figuren/tabellen altijd een korte beschrijving:
%
%  \caption[korte beschrijving]{uitgebreide beschrijving}
%
% De korte beschrijving wordt gebruikt voor deze lijst, de uitgebreide staat bij
% de figuur of tabel zelf.

\listoffigures
\listoftables

% Als je een lijst van afkortingen of termen wil toevoegen, dan hoort die
% hier thuis. Gebruik bijvoorbeeld de ``glossaries'' package.
% https://www.overleaf.com/learn/latex/Glossaries

%---------- Kern ---------------------------------------------------------------

% De eerste hoofdstukken van een bachelorproef zijn meestal een inleiding op
% het onderwerp, literatuurstudie en verantwoording methodologie.
% Aarzel niet om een meer beschrijvende titel aan deze hoofstukken te geven of
% om bijvoorbeeld de inleiding en/of stand van zaken over meerdere hoofdstukken
% te verspreiden!
 
%%=============================================================================
%% Inleiding
%%=============================================================================

\chapter{\IfLanguageName{dutch}{Inleiding}{Introduction}}
\label{ch:inleiding}

De inleiding moet de lezer net genoeg informatie verschaffen om het onderwerp te begrijpen en in te zien waarom de onderzoeksvraag de moeite waard is om te onderzoeken. In de inleiding ga je literatuurverwijzingen beperken, zodat de tekst vlot leesbaar blijft. Je kan de inleiding verder onderverdelen in secties als dit de tekst verduidelijkt. Zaken die aan bod kunnen komen in de inleiding~\autocite{Pollefliet2011}:

\begin{itemize}
  \item context, achtergrond
  \item afbakenen van het onderwerp
  \item verantwoording van het onderwerp, methodologie
  \item probleemstelling
  \item onderzoeksdoelstelling
  \item onderzoeksvraag
  \item \ldots
\end{itemize}

\section{\IfLanguageName{dutch}{Probleemstelling}{Problem Statement}}
\label{sec:probleemstelling}

Uit je probleemstelling moet duidelijk zijn dat je onderzoek een meerwaarde heeft voor een concrete doelgroep. De doelgroep moet goed gedefinieerd en afgelijnd zijn. Doelgroepen als ``bedrijven,'' ``KMO's,'' systeembeheerders, enz.~zijn nog te vaag. Als je een lijstje kan maken van de personen/organisaties die een meerwaarde zullen vinden in deze bachelorproef (dit is eigenlijk je steekproefkader), dan is dat een indicatie dat de doelgroep goed gedefinieerd is. Dit kan een enkel bedrijf zijn of zelfs één persoon (je co-promotor/opdrachtgever).

\section{\IfLanguageName{dutch}{Onderzoeksvraag}{Research question}}
\label{sec:onderzoeksvraag}

Wees zo concreet mogelijk bij het formuleren van je onderzoeksvraag. Een onderzoeksvraag is trouwens iets waar nog niemand op dit moment een antwoord heeft (voor zover je kan nagaan). Het opzoeken van bestaande informatie (bv. ``welke tools bestaan er voor deze toepassing?'') is dus geen onderzoeksvraag. Je kan de onderzoeksvraag verder specifiëren in deelvragen. Bv.~als je onderzoek gaat over performantiemetingen, dan 

\section{\IfLanguageName{dutch}{Onderzoeksdoelstelling}{Research objective}}
\label{sec:onderzoeksdoelstelling}

Wat is het beoogde resultaat van je bachelorproef? Wat zijn de criteria voor succes? Beschrijf die zo concreet mogelijk. Gaat het bv. om een proof-of-concept, een prototype, een verslag met aanbevelingen, een vergelijkende studie, enz.

\section{\IfLanguageName{dutch}{Opzet van deze bachelorproef}{Structure of this bachelor thesis}}
\label{sec:opzet-bachelorproef}

% Het is gebruikelijk aan het einde van de inleiding een overzicht te
% geven van de opbouw van de rest van de tekst. Deze sectie bevat al een aanzet
% die je kan aanvullen/aanpassen in functie van je eigen tekst.

De rest van deze bachelorproef is als volgt opgebouwd:

In Hoofdstuk~\ref{ch:stand-van-zaken} wordt een overzicht gegeven van de stand van zaken binnen het onderzoeksdomein, op basis van een literatuurstudie.

In Hoofdstuk~\ref{ch:methodologie} wordt de methodologie toegelicht en worden de gebruikte onderzoekstechnieken besproken om een antwoord te kunnen formuleren op de onderzoeksvragen.

% TODO: Vul hier aan voor je eigen hoofdstukken, één of twee zinnen per hoofdstuk

In Hoofdstuk~\ref{ch:conclusie}, tenslotte, wordt de conclusie gegeven en een antwoord geformuleerd op de onderzoeksvragen. Daarbij wordt ook een aanzet gegeven voor toekomstig onderzoek binnen dit domein.
\chapter{\IfLanguageName{dutch}{Stand van zaken}{State of the art}}
\label{ch:stand-van-zaken}

Dit onderzoek is een vergelijkende studie tussen verschillende PostgreSQL oplossingen. In dit hoofdstuk zullen al verschillende oplossingen aan bod komen. Hiermee zal worden gekeken naar de nadruk die gelegd wordt bij elke oplossing. Op deze manier kan gekeken worden naar welke elementen regelmatig terugkomen, en dus welke we kunnen gebruiken om verschillende oplossingen met elkaar te vergelijken.

% Tip: Begin elk hoofdstuk met een paragraaf inleiding die beschrijft hoe
% dit hoofdstuk past binnen het geheel van de bachelorproef. Geef in het
% bijzonder aan wat de link is met het vorige en volgende hoofdstuk.
% Pas na deze inleidende paragraaf komt de eerste sectiehoofding.
%Dit hoofdstuk bevat je literatuurstudie. De inhoud gaat verder op de inleiding, maar zal het onderwerp van de bachelorproef *diepgaand* uitspitten. De bedoeling is dat de lezer na lezing van dit hoofdstuk helemaal op de hoogte is van de huidige stand van zaken (state-of-the-art) in het onderzoeksdomein. Iemand die niet vertrouwd is met het onderwerp, weet nu voldoende om de rest van het verhaal te kunnen volgen, zonder dat die er nog andere informatie moet over opzoeken \autocite{Pollefliet2011}.
%Je verwijst bij elke bewering die je doet, vakterm die je introduceert, enz. naar je bronnen. In \LaTeX{} kan dat met het commando \texttt{$\backslash${textcite\{\}}} of \texttt{$\backslash${autocite\{\}}}. Als argument van het commando geef je de ``sleutel'' van een ``record'' in een bibliografische databank in het Bib\LaTeX{}-formaat (een tekstbestand). Als je expliciet naar de auteur verwijst in de zin, gebruik je \texttt{$\backslash${}textcite\{\}}.
%Soms wil je de auteur niet expliciet vernoemen, dan gebruik je \texttt{$\backslash${}autocite\{\}}. In de volgende paragraaf een voorbeeld van elk.
%\textcite{Knuth1998} schreef een van de standaardwerken over sorteer- en zoekalgoritmen. Experten zijn het erover eens dat cloud computing een interessante opportuniteit vormen, zowel voor gebruikers als voor dienstverleners op vlak van informatietechnologie~\autocite{Creeger2009}.

\section{\IfLanguageName{dutch}{PosgreSQL}{PosgreSQL}}
\label{sec:PosgreSQL}

\textbf{Databank}

Wikipedia omschrijft een databank als een georganiseerde collectie van data, opgeslagen in en toegankelijk vanuit een computersysteem %(https://en.wikipedia.org/wiki/Database#:~:text=A\%20database\%20is\%20an\%20organized,formal\%20design\%20and\%20modeling\%20techniques.)



\textbf{Relationele databank}

Een relationele databank is een type databank waarin gebruik wordt gemaakt van een structuur die het mogelijk maakt om gegevens te identificeren en te benaderen in relatie tot een ander deeltje data in diezelfde databank. Deze gegevens worden vaak georganiseerd in tabellen. Deze tabellen kunnen honderden, duizenden, miljoenen rijen en kolommen aan data hebben. Een kolom kent vaak ook een specifiek gegevenstype. Deze gegevenstypes kunnen getallen (integers), woorden (strings) of andere soorten bevatten. %(https://www.codecademy.com/articles/what-is-rdbms-sql#:~:text=A\%20relational\%20database\%20is\%20a,database\%20is\%20organized\%20into\%20tables.)



\textbf{Object-georiënteerde databank}

Een objectgeoriënteerde database (OODBMS) is een type databank die zich baseert op objectgeoriënteerd programmeren (OOP). De gegevens worden hier voorgesteld en opgeslagen in de vorm van objecten. OODBMS worden ook objectdatabases of objectgeoriënteerde databasemanagementsystemen genoemd %(https://www.c-sharpcorner.com/article/what-are-object-oriented-databases-and-their-advantages2/#:~:text=An%20object%2Doriented%20database%20(OODBMS,object%2Doriented%20database%20management%20systems.) 


\textbf{SQL}

SQL (structured query language) is de eigen taal specifiek ontwikkelt voor interactie met databanken. Een databank modelleert entiteiten uit het echte leven en slaat deze op in tabellen. Via SQL is het mogelijk om de gegevens in deze tabellen te manipuleren.  %(https://www.datacamp.com/community/tutorials/what-is-sql?utm_source=adwords_ppc&utm_campaignid=898687156&utm_adgroupid=48947256715&utm_device=c&utm_keyword=&utm_matchtype=b&utm_network=g&utm_adpostion=&utm_creative=229765585183&utm_targetid=dsa-429603003980&utm_loc_interest_ms=&utm_loc_physical_ms=1001208&gclid=Cj0KCQjwse-DBhC7ARIsAI8YcWJOxNwWfRDfjloPnAcswV6QsafwgHI0hqa-gQFH5fj59JLPM04NOxMaAmudEALw_wcB)


\textbf{Object-relationele databank}

Een object-relationele database (ORD / ORDBMS) is een samenstelling uit zowel een relationele database (RDB / RDBMS), als een object-georiënteerde database (OOD / OODBMS). Samen ondersteunt het de basiscomponenten van elk objectgeoriënteerd databasemodel in zijn schema's en gebruikte querytaal, zoals klassen, overerving en objecten. Het bevat aspecten en kenmerken van bovenstaande genoemde modellen. Zo wordt het relationele duidelijk in de manier van opslaan van gegevens. Deze worden opgeslagen in een traditionele database en worden dan met behulp van SQL query's gemanipuleerd en benaderd. Aan de andere kant is ook het objectgeoriënteerde gedeelte merkbaar, namelijk dat de database beschouwd wordt als een objectopslag. Kort gezegd is één van de voornaamste doelstellingen van een objecte-relationele database, het dichten van de kloof tussen relationele en objectgeoriënteerde modelleringstechnieken en conceptuele datamodelleringstechnieken zoals daar zijn het entiteit-relatiediagram (ERD) en object-relationeel mappen (ORM) (https://www.techopedia.com/definition/8714/object-relational-database-ord).
%Hieronder bij klasses en overerving kan er al een foto bij ter verduidelijking.
Klasses, Overerving, Types, Functies zijn kenmerken van de object-relationele database en zullen de basisconcepten vormen voor PostgreSQL. Een klasse is een verzameling van gegevenstypes die bij eenzelfde soort iets horen. Bijvoorbeeld een klasse CD kan als kenmerken hebben: Titel, zanger, Datum uitgave, aantal liedjes... . Overerving is wanneer een klasse bepaalde kenmerken overerft, krijgt van een superklasse. Bijvoorbeeld een klasse Olifant erft van de klasse Zoogdier kenmerken. Hierin is de klasse Olifant een specialisatie van de klasse Zoogdier. Een type is hierboven al eens genoemd geweest. Dit gaat over de verschillende soorten data die er zijn. Getallen, woorden, objecten zijn hier voorbeelden van. Functies zijn een reeks SQL-statements die een specifieke taak uitvoeren. Functies bevorderen de herbruikbaarheid van code.



\textbf{PostgreSQL}

PostgreSQL is een open source systeem dat zich toelegt op het beheer van object-relationele databases. Het heeft meer dan 30 jaar actieve ontwikkeling en heeft een sterke reputatie op vlak van betrouwbaarheid, robuustheid van functies en prestaties (https://www.postgresql.org/). PostgreSQL biedt een uitgebreide set van functionaliteiten die een hoge mate van customisatie mogelijk maakt binnen het systeem. Dit gaat van data administratie, beveiliging, tot backup en herstel . PostgreSQL wordt regelmatig bijgewerkt door de PostgreSQL Global Development Group en bijdragers uit de community. Deze community ondersteunt zichzelf en zijn gebruikers door het aanbieden van online educatieve bronnen en communicatiekanalen, zoals daar zijn PostgreSQL wiki, online forums en officiële documentatie. Er zijn ook bedrijven die commerciële support bieden aan een prijs (https://nethosting.com/mysql-vs-postgresql-2019-showdown/). Volgens DB-Engines is PostgreSQL de vierde database die vandaag de dag het meest gebruikt wordt en de tweede meest gebruikte open source database, na MySQL (https://db-engines.com/en/ranking). DB-Engines verklaarde PostgreSQL in 2017, 2018 en 2020 het DBMS (Database management system) van het jaar (https://db-engines.com/en/system/PostgreSQL). PostgreSQL biedt veel mogelijkheden om ontwikkelaars te helpen bij het bouwen van applicaties, om beheerders te helpen bij het beschermen van data-integriteit en het bouwen van fouttolerante omgevingen. Het helpt ook bij het beheren van data, hoe groot of hoe klein de dataset ook is. PostgreSQL voldoet sinds september 2020 aan 170 van de 179 verplichte functies voor SQL:2016 Core conformiteit. Schaalbaarheid valt ook toe te schrijven aan PostgreSQL, dit zowel in de hoeveelheid data die het kan beheren, als in het aantal gelijktijdige gebruikers dat het kan accomoderen. Er zijn actieve PostgreSQL clusters in productie omgevingen die terabytes aan data beheren, en gespecialiseerde systemen die petabytes beheren (https://www.postgresql.org/about/).

Postgres speelt in op de bovenvernoemde vier basisconcepten van een object-relationele databank zodat gebruikers het systeem makkelijk kunnen uitbreiden. Deze vier kenmerken, naast nog andere functies maken van Postgres een object-relationele database. Bovenstaande vermelde kenmerken zouden doen blijken dat Postgres voornamelijk een object-georiënteerde database is, maar de ondersteuning van de traditionele relationele databases, toont duidelijk aan dat, ondanks de object-georiënteerde kenmerken, Postgres stevig verankerd is in de relationele database wereld (https://www.postgresql.org/docs/6.3/c0101.htm).







%Hoe lost PostgreSQL bepaalde problemen op




%Een object-relationele database (ORD / ORDBMS) is eigenlijk een samenstelling uit zowel een relationele database (RDB / RDBMS), als een object-georiënteerde database (OOD / OODBMS). Samen ondersteunt het de basiscomponenten van elk objectgeoriënteerd databasemodel in zijn schema's en gebruikte querytaal, zoals klassen, overerving een objecten. Het bevat aspecten en kenmerken van bovenstaande genoemde modellen. Zo zien we het relationele in de manier van het opslaan van gegevens. Deze worden opgeslagen in een traditionele database en worden dan met behulp van query's, zoals SQL, gemanipuleerd en benaderd. Aan de andere kant zien we ook het objectgeoriënteerde gedeelte, namelijk dat de database beschouwd wordt als een objectopslag. Kort gezegd is één van de voornaamste doelstellingen van een objecte-relationele database, het dichten van de kloof tussen relationele en objectgeoriënteerde modelleringstechnieken en conceptuele datamodelleringstechnieken zoals daar zijn het entiteit-relatiediagram (ERD) en object-relationeel mappen (ORM) (https://www.techopedia.com/definition/8714/object-relational-database-ord).
%Postgres wil hierop inspelen door het inbouwen van volgende vier basisconcepten die ervoor zorgen dat gebruikers het systeem gemakkelijk kunnen uitbreiden:

%Klasses, Overerving, Types, Functies

%Deze vier kenmerken, naast nog andere functies maken van Postgres een object-relationele database. Bovenstaande vermelde kenmerken zouden doen blijken dat Postgres voornamelijk een object-georiënteerde database is, maar de ondersteuning van de traditionele relationele databases, toont ons duidelijk dat, ondanks de object-georiënteerde kenmerken, Postgres stevig verankerd is in de relationele database wereld (https://www.postgresql.org/docs/6.3/c0101.htm).



\section{\IfLanguageName{dutch}{High Availability}{High Availability}}
\label{sec:High Availability}

Het doel van High Availability architectuur is ervoor te zorgen dat een server, website of applicatie verschillende vraagbelastingen en verschillende soorten storingen kan verdragen. En dit met de minst mogelijke downtime. Door gebruik te maken van best practices die zijn ontworpen om hoge beschikbaarheid te garanderen, helpt dit volgens ServersAustralia om in een organisatie maximale productiviteit en betrouwbaarheid te bereiken. (https://www.serversaustralia.com.au/resources/blog/what-is-high-availability-ha-and-do-i-need-it/#:~:text=The%20purpose%20of%20HA%20architecture,achieve%20maximum%20productivity%20and%20reliability.)
% Waarom is dit nodig?
% Wat zijn de mogelijkheden om hieraan te voldoen? en Hoe implementeert postgres deze oplossingen

High Availability is het vermogen van een systeem om continu operationeel te blijven te zijn gedurende een wenselijk lange tijd. Men kan Availability meten ten opzicht van 100\% operationeel, als in, nooit uitvallen. Beschikbaarheid wordt vaak uitgedrukt als een percentage van uptime in een bepaald jaar op basis van de SLA's, Service Level Agreements (https://www.enterprisedb.com/blog/what-does-database-high-availability-really-mean). Vaak duidt men deze norm aan als de Five 9's, namelijk 99,999\% beschikbaarheid (https://searchdatacenter.techtarget.com/definition/high-availability). High availability impliceert dat delen van een systeem volledig zijn getest en dat er voorzieningen zijn voor storingen/failures in de vorm van redundante componenten. Servers kunnen worden ingesteld om in geval van nood de verantwoordelijkheden over te dragen aan een externe server, in een back-up proces. Hier spreekt men dan van failover (https://www.techopedia.com/definition/1021/high-availability-ha).

% https://www.postgresql.org/docs/9.5/high-availability.html Deze link nog eens uitspitten

Belangrijke principes van High Availability zijn:

1. \textbf{Het elimineren van single point of failure}: Toevoeging van redundantie zorgt er voor zodat het falen van een onderdeel in het systeem niet leidt tot het volledige falen van een geheel systeem.

2. \textbf{Betrouwbare cross-over}: In een redundant systeem wordt het kruispunt zelf een single point of failure. Fouttolerante systemen moeten voorzien in een betrouwbaar crossover- of automatisch omschakelingsmechanisme om storingen te voorkomen.

3. \textbf{Storingdetectie}: Als bovenstaande principes proactief bewaakt worden, dan zal een gebruik misschien nooit een systeemstoring zien.
Postgres biedt de bouwstenen om bovenstaande principes volledig uit te werken zodat er op deze manier High Availability verzekerd kan worden.

% ### Dit kan ook nog bij Cluster en oplossingen gezet worden

Bij het elimineren van single points of failure ondersteunt Postgres de volgende fysieke stand-by's:

1. \textbf{Cold Standby}: Dit is een back-up server die beschikt over back-ups en alle nodige WAL-bestanden voor herstel. WAL is de afkorting voor Write Ahead Log. Het logt elke transactie die uitgevoerd wordt op een database voordat het wordt uitgevoerd. Een Cold Standby systeem is geen operationeel systeem, maar het kan wel beschikbaar worden gemaakt als dat nodig is. Voornamelijk worden dan backup servers en WAL bestanden gebruikt voor het maken van een nieuwe PostgreSQL node als onderdeel van disaster recovery.

2. \textbf{Warm Standby}: Hierin draait Postgres in herstelmodus en ontvangt updates door gebruik te maken van gearchiveerde logbestanden of door gebruik te maken van log shipping replicatie van Postgres. Log shipping is een proces waarbij de back-up van transactielogbestanden op een primaire database wordt geautomatiseerd en vervolgens op een standby server wordt hersteld (https://docs.microsoft.com/en-us/sql/database-engine/log-shipping/about-log-shipping-sql-server?view=sql-server-ver15). In deze modus aanvaardt Postgres geen verbindingen en queries.

3. \textbf{Hot Standby}: Ook bij Hot Standby draait Postgres in herstelmodus en ontvangt het updates door gebruik te maken van gearchiveerde logbestanden of door gebruik te maken van log shipping van Postgres. Het verschil met Warm Standby is dat in deze herstelmodus Postgres hier wel verbindingen ondersteunt en read-only queries.

Bovenstaande voorbeelden zijn mogelijkheden die kunnen helpen bij het elimineren van single points of failure. Afhankelijk van het overeengekomen niveau van beschikbaarheid, kunnen gebruikers voor een van de bovenstaande kiezen.

In geval van een volledige uitval van een systeem is geografische redundantie algemeen zeer wenselijk. Op deze manier worden servers verdeeld over meerdere locaties verdeeld over de wereld. Bij downtime door een natuurramp bijvoorbeeld zijn standby servers op meerdere fysieke (ongetroffen) locaties beschikbaar om in te vallen. Dit type van redundantie kan zeer duur uitdraaien, waarbij het een verstandige beslissing kan zijn om te kiezen voor een gehoste oplossing, waarbij de provider datacenters heeft over heel de wereld.


\textbf{Load Balancing}

Load Balancing is ook een manier om High Availability te waarborgen. Het doel van een load balancer is om toepassingen en/of netwerkverkeer te verdelen over meerdere servers en componenten. Het zal binnenkomende verzoeken routeren naar verschillende servers. Hiermeer wil het optionele prestaties en betrouwbaarheid verbeteren. Enkele voorbeelden van load balancing is Round Robin die ervoor zorgt dat de verzoeken van de load balancer naar de eerste server gaan in de rij. De verzoeken gaan deze rij af, tot hij op het einde komt, waarna hij terug van het eerste element in de rij begint. Een tweede manier van load balancing is Least Connection. Hierbij zal er gekozen worden om gebruik te maken van de server met het minst aantal actieve verbindingen. Load balancers spelen een rol bij het tot stand brengen van een infrastructuur met High Availability, maar het hebben van een load balancer staat niet garantie voor het hebben van High Availability. Door redundantie te implementeren voor de load balancer zelf, kan deze geelimineerd worden als een single point of failure.
(https://phoenixnap.com/blog/what-is-high-availability)



\section{\IfLanguageName{dutch}{Cluster oplossingen}{Cluster solutions}}
\label{sec:Cluster solutions}
% de tools die HA mogelijkheden naar een hoger niveau tillen

\textbf{Cluster}

Een cluster is een groepering van servers die met elkaar samenwerken om één geheel te vormen. Op deze manier kan een cluster High Availability mogelijk maken.(https://whatis.techtarget.com/definition/cluster)



Failover is het automatisch overschakelen naar een back-upsysteem. Wanneer een primair systeemonderdeel faalt, wordt failover ingeschakeld om de negatieve gevolgen te elimineren of te beperken. (https://avinetworks.com/glossary/failover/)

Failback is het proces van het herstellen van operaties naar een primaire machine  nadat ze zijn verschoven geweest naar een secundaire machine wegens failover. (https://whatis.techtarget.com/definition/failback)


\textbf{Patroni}

%https://www.cybertec-postgresql.com/en/services/postgresql-replication/high-availability-patroni/

Patroni is een open source cluster-technologie die zich bezighoudt met automatische failover en High Availability voor een PostgreSQL databank. Het dient als een soort cluster manager die de implementatie en het onderhoud van High Availability in PostgreSQl clusters zal aanpassen en automatiseren. Het maakt gebruik van gedistribueerde configuratieopslagplaatsen zoals etcd, Consul, ZooKeeper of Kubernetes voor maximale toegankelijkheid.
(https://www.cybertec-postgresql.com/en/patroni-setting-up-a-highly-available-postgresql-cluster/)

Patroni biedt cloud-native netwerkfuncties en geavanceerde opties voor failback en failover. Een cloud-native netwerkfunctie is een software-implementatie van een netwerkfunctie, die wordt uitgevoerd in een linux-container, die traditioneel wordt uitgevoerd door een fysiek apparaat %(https://cdnf.io/what_is_cnf/).


https://buildmedia.readthedocs.org/media/pdf/patroni/latest/patroni.pdf
https://blog.dbi-services.com/postgresql-high-availabilty-patroni-ectd-haproxy-keepalived/


\textbf{Pgpool-II}

%https://www.pgpool.net/mediawiki/index.php/Main_Page

Pgpool-II omschrijft zichzelf als een middleware die werkt tussen PostgreSQL servers en een PostgreSQL database client. Pgpool-II biedt de volgende features:

1. Connection Pooling

Pgpool-II bewaart verbindingen naar de PostgreSQL servers, en hergebruikt ze wanneer een nieuwe verbinding met dezelfde eigenschappen zoals gebruikersnaam, database of protocol versie binnenkomt. Dit vermindert de overhead van verbindingen, en verbetert de totale doorvoer van het systeem.

2. Replication

Pgpool-II kan meerdere PostgreSQL servers beheren. Door gebruik te maken van de replicatiefunctie kan een realtime backup worden gemaakt op 2 of meerdere fysieke schijven, zodat de dienst kan worden voortgezet zonder servers te stoppen in geval van een schijfstoring.

3. Load Balancing

Als een database wordt gerepliceerd, zal het uitvoeren van een query op elke server hetzelfde resultaat opleveren. Pgpool-II maakt gebruik van de replicatie mogelijkheid om de belasting op elke PostgreSQL server te verminderen door queries over meerdere servers te verdelen, waardoor de totale throughput van het systeem verbetert. In het meest gunstige geval verbetert de prestatie evenredig met het aantal PostgreSQL servers. Load balance werkt het beste in een situatie waarin er veel gebruikers zijn die veel queries tegelijkertijd uitvoeren.

4. Limiting Exceeding Connections

Er is een limiet op het maximum aantal gelijktijdige verbindingen met PostgreSQL, en verbindingen worden geweigerd na dit maximaum aantal verbindingen. Het instellen van een max aantal verbindingen verhoogt echter het verbruik van bronnen en beïnvloedt de systeemprestaties. pgpool-II heeft ook een limiet op het maximum aantal verbindingen, maar extra verbindingen worden dan in een wachtrij geplaatst..

5. Watchdog

Watchdog kan een robuust clustersysteem creëren en het single point of failure of split brain vermijden. Watchdog kan een lifecheck uitvoeren tegen andere Pgpool-II nodes, om een fout van Pgpool-II te detecteren. Als actieve Pgpool-II down gaat, kan dan een standby Pgpool-II gepromoveerd worden tot actief, en zal deze dan virtueel het IP overnemen.

6. In Memory Query Cache

In memory query cache maakt het mogelijk om een paar SELECT statements en zijn resultaat op te slaan. Als een identieke SELECT binnenkomt, retourneert Pgpool-II de waarde uit de cache. Omdat er geen SQL parsing of toegang tot PostgreSQL aan te pas komt, is het gebruik van in memory cache extreem snel. Aan de andere kant kan het in sommige gevallen langzamer zijn dan het normale pad, omdat het wat overhead toevoegt van het opslaan van cache gegevens.

Pgpool-II praat met de backend en de frontend protocollen van PostgreSQL, en legt een verbinding tussen beide. Daarom denkt een database applicatie (frontend) dat Pgpool-II de eigenlijke PostgreSQL server is, en de server (backend) ziet Pgpool-II als een van zijn clients. Omdat Pgpool-II transparant is voor zowel de server als de client, kan een bestaande databasetoepassing met Pgpool-II worden gebruikt vrijwel zonder de broncode aan te passen. %(https://www.pgpool.net/mediawiki/index.php/Main_Page)



\textbf{PostgreSQL Automatic Failover (PAF)}





\textbf{RepMgr [Replication Manager]}

%https://wiki.postgresql.org/wiki/Repmgr#repmgr_5_Features

Repmgr is een open-source oplossing die zich bezighoudt met het beheren van replicatie en failover van servers in PostgreSQl clusters. Het verbetert de ingebouwde hot-standby opties van PostgreSQL met extra features zoals tools om standby servers op te zetten, replicatie te monitoren en administratieve taken uit te voeren, zoals failover. (https://repmgr.org/)


De features die repmgr 5 aanbiedt zijn:

1. De implementatie als een PostgreSQL extentie.

2. Replicatie cluster monitoring.

3. Standby klonen aan de hand van pg\_basebackup of Barman

pg\_basebackup: pg\_basebackup wordt gebruikt om basisbackups te maken van een draaiende PostgreSQL databank cluster. Deze back-ups worden gemaakt zonder de aanwezige cliënts, die in verbinding staan met de databank, te beïnvloeden. Deze back-ups kunnen gebruikt worden voor zowel point-in-time recovery, maar ook als een startpunt voor log shipping of streaming replicatie standby servers. Bij point-in-time recovery wordt verwezen naar herstel van dataveranderingen tot een bepaald punt in de tijd. %(https://dev.mysql.com/doc/mysql-backup-excerpt/8.0/en/point-in-time-recovery.html#:~:text=Point%2Din%2Dtime%20recovery%20refers,time%20the%20backup%20was%20made.)

pg\_basebackup maakt een binaire kopie van de database cluster bestanden, terwijl het ervoor zorgt dat het systeem automatisch in en uit backup modus wordt gezet. Backups worden altijd gemaakt van de gehele databasecluster; het is niet mogelijk om een backup te maken van afzonderlijke databases of databaseobjecten. 

De backup wordt gemaakt over een gewone PostgreSQL verbinding maakt gebruik van het replicatieprotocol. De server moet ook worden geconfigureerd om ten minste één sessie beschikbaar te laten voor de backup. (https://www.postgresql.org/docs/10/app-pgbasebackup.html)

Barman: Barman of pgbarman staat voor Backup en Recovery Manager. Het is een open-source beheertool voor disaster recovery van PostgreSQL servers. Het is geschreven in Python. Barman laat toe om van op afstand van meerdere servers in bedrijfskritische omgevingen back-ups uit te voeren. Het helpt ook database beheerders tijdens een herstelfase. (https://www.pgbarman.org/about/)(https://www.pgbarman.org/)

4. Standby server die kan worden gepromoveerd tot een primary server zonder herstart. Andere standby servers die verbinding kunnen maken met de nieuwe master zonder opnieuw gesynchroniseerd te worden.
 .%https://wiki.postgresql.org/wiki/Repmgr#repmgr_5_Features

5.Cascading Standby Support

Standby servers die niet direct verbonden zijn met de master node worden niet beïnvloed tijdens failover van de primary naar een andere standby mode.


6. Vereenvoudigen van het beheer van WAL-retentie en ondersteuning voor repliatiesleuven

Door de vereenvoudiging van WAL-retentie zal het dus eenvoudiger zijn om WAL-bestanden op te ruimen vanaf elke bestandssysteemlocatie. Ook in standby servers kan het gebruikt worden om bestanden die niet meer nodig zijn, te verwijderen uit de standby server.
%(https://www.percona.com/blog/2019/07/10/wal-retention-and-clean-up-pg_archivecleanup/)

7. Switchover ondersteuning voor rolswitching tussen primary en standby

Hierin wordt een standby server geprovomeerd tot een primary server en zal deze primary server degraderen naar een standby server. Wanneer andere standby servers verbonden zijn met de degradatiekandidaat, kan repmgr deze instrueren om de nieuwe primary server te volgen en niet de oude, die net gedegradeerd is. (https://repmgr.org/docs/4.0/repmgr-standby-switchover.html)


\textbf{PgCluster}



\textbf{Raima}

Databasereplicatie is het proces waarin we gegevens gaan kopiëren van een database naar één of meerdere replica's. Dit om de toegankelijkheid van gegevens en fouttolerantie te verbeteren.
In de context van replicatie gebruikt men vaak ook de termen actief-actief en actief-passief. Raima Database Manager (RDM) ondersteunt beide technieken. Bij Raima wordt actieve replicatie gewoon replicatie genoemd en passieve replicatie spiegelen (mirroring). Spiegelen zal resulteren in identieke replica's zoals de originele database, terwijl replicatie zal resulteren in replica's die niet identiek zijn aan de originele database. Deze replica's zullen alle records bevatten die van de originele database zijn overgebracht, maar de fysieke organisatie van de records in de databasebestanden (of in het geheugen) kan verschillen.
Om terug te komen op de termen actief-actief en actief-passief zullen die vaker verwijzen naar andere concepten dan dewelke we juist omschreven hebben (replicatie en spiegelen). Actieve-actieve replicatie betekent replicatie in twee richtingen van gegevens tussen twee databases die beide actief worden bijgewerkt. Actieve-passieve replicatie betekent replicatie in één richting van een actief bijgewerkte master node naar een slave node die niet wordt bijgewerkt, behalve door het replicatieproces. Hier verwijst men soms ook naar master-slave replicatie. In RDM is replicatie altijd actief-passief (https://raima.com/rdme-high-availability-database/).

% ### Raima link verder uitwerken


\textbf{EDB}

Voor betrouwbare cross-over biedt EDB een technologie genaamd EDB Postgres Failover Manager (EFM). Dit maakt automatische failover van de Postgres master node naar een standby node mogelijk in geval van een software- of hardwarefout op de master. EFM maakt gebruik van JGroups, die een betrouwbare, gedistribueerde en redundante infrastructuur biedt zonder een single point of failure.
EDB Postgres Failover Manager kan ook gebruikt worden voor de detectie van storingen. Het bewaakt de server continu en zal storingen op verschillende niveaus detecteren. Het is ook capabel om om failover uit te voeren van de master node naar één van de replica nodes om het systeem beschikbaar te maken voor het accepteren van databaseverbindingen en queries. Wanneer EFM goed geconfigureerd is, kan het storingen detecteren en direct failover uitvoeren.

Verlies van service kunnen we in twee categoriëen opdelen. Geplande uitval of downtime en ongeplande uitval of downtime.
Geplande downtime is vaak het gevolg van onderhoudsactiviteiten. Dit kan zijn door een softwarepatches die een herstart van het systeem of van de database vereist. In het algemeen is deze uitval niet onverwachts en zal deze uitval geen grootschalige gevolgen hebben.
Een ongeplande downtime is vaak het resultaat van een of andere fysieke gebeurtenis, zoals hardware- of softwarestoring, of een anomalie in de omgeving. Stroomuitval, defecte CPU- of RAM-componenten (of eventueel andere hardwarecomponenten), netwerkstoringen, inbreuken op de beveiliging, of diverse defecten in toepassingen, middleware en besturingssystemen resulteren bijvoorbeeld in ongeplande uitval.
In geval van (on)geplande downtime kan EFM helpen om de downtime zoveel mogelijk te minimaliseren. Voor een geplande downtime kan een gebruiker bijvoorbeeld eerst alle standby nodes patchen en EFM gebruiken om over te schakelen voordat de master node gepatcht wordt. Bij een ongeplande downtime kan EFM ervoor zorgen dat de storingen gedetecteerd worden en failover uitvoeren naar de juiste standby node, om dan deze node de nieuwe master node te maken. EFM zal na dit proces er ook voor zorgen dat de oude master node niet terugkomt om een split-brain situatie te voorkomen. Split-brain duidt op de inconsistenties in beschikbaarheid en data. Hierdoor ontstaan er twee afzonderlijke datasets met overlap. (https://www.enterprisedb.com/blog/what-does-database-high-availability-really-mean).


\textbf{Slony}



\textbf{Bucardo}



\textbf{Londiste}



\textbf{Mammoth}



\textbf{Rubyrep}



\textbf{pg\_shard}



\textbf{pgLogical}



\textbf{Postgres-XL}



\textbf{Citus}

https://www.citusdata.com/blog/2019/05/30/introducing-pg-auto-failover/



% (https://wiki.postgresql.org/wiki/Replication,_Clustering,_and_Connection_Pooling)
% (https://wiki.postgresql.org/wiki/Clustering)



Van elk zal ik dan kijken of ze open source zijn (normaal gezien zijn ze dit allemaal al, maar het kan geen kwaad om dit nog eens op te zoeken), welke focus ze leggen...






\section{\IfLanguageName{dutch}{Puppet}{Puppet}}
\label{sec:Puppet}

Puppet CTO Deepak Giridharagopal zei dat in het kielzog van de economische neergang als gevolg van de COVID-19 pandemie, meer IT-teams zwaar zullen moeten vertrouwen op automatisering. De meeste IT-teams zullen ofwel even groot blijven of worden ingekrompen. De IT-omgeving zal echter steeds complexer worden. De enige manier om IT-teams in staat te stellen meer te doen met minder is het automatiseren van meer routinetaken (https://devops.com/puppet-brings-orchestration-to-it-automation/).

Puppet is een cross-platform client-server gebaseerde toepassing die wordt gebruikt voor configuratiebeheer. Het behandelt de software en zijn configuraties op meerdere servers. Er zijn hierbij twee versies beschikbaar. De ene is open-source, de andere is een betalende, commerciële versie. Het werkt op zowel Linux als op Windows. Het gebruikt een declaratieve aanpak om updates, installaties en andere taken te automatiseren. De software kan systemen configureren met behulp van bestanden die manifesten worden genoemd. Een manifest bevat instructies voor een groep of type server(s) die wordt/worden beheerd(https://www.liquidweb.com/kb/what-is-puppet-and-what-role-does-it-play-in-devops/). 

Wat is configuratiebeheer nu juist? Configuratiebeheer onderhoudt en bepaalt productkenmerken door fysieke en functionele attributen, ontwerp, vereisten en operationele informatie op te slaan gedurende de levenscyclus van een server. 

Puppet maakt gebruik van de beschrijvende programmeertaal Ruby. Ruby is een dynamische, open source programmeertaal met de nadruk op eenvoud en productiviteit (https://www.ruby-lang.org/en/).

Vroeger werden software en systemen door systeembeheerders manueel opgezet en geconfigureerd. Maar toen het te beheren aantal servers snel toenam, moest er gezocht worden naar een manier om die processen te automatiseren, om dan zo tijd te besparen en de nauwkeurigheid te vergroten. Puppet is uit deze zoektocht ontstaan.

Puppet werkt aan de hand van een eenvoudig client/server architectuur workflow proces. Hierin bestaat er een master server die alle informatie bevat over de configuraties van de verschillende nodes aanwezig. Het slaat deze configuraties op in manifestbestanden op een een centrale server, genaamd de Puppet master, en voert deze manifesten uit op de remote client servers genaamd agents (https://www.liquidweb.com/kb/what-is-puppet-and-what-role-does-it-play-in-devops/).


%%=============================================================================
%% Methodologie
%%=============================================================================

\chapter{\IfLanguageName{dutch}{Methodologie}{Methodology}}
\label{ch:methodologie}

%% TODO: Hoe ben je te werk gegaan? Verdeel je onderzoek in grote fasen, en
%% licht in elke fase toe welke stappen je gevolgd hebt. Verantwoord waarom je
%% op deze manier te werk gegaan bent. Je moet kunnen aantonen dat je de best
%% mogelijke manier toegepast hebt om een antwoord te vinden op de
%% onderzoeksvraag.

Zoals in vele andere onderzoeken ook het geval is, is dit onderzoek gestart met een diepgaande en extensieve literatuurstudie over PostgreSQL, High Availability, Puppet en de reeds bestaande High Available PostgreSQL cluster oplossingen. Deze literatuurstudie is terug te vinden in Hoofdstuk 2: Stand van zaken.

Na de literatuurstudie zal worden geduid hoe de verschillende High Available PostgreSQL cluster oplossingen geschift zullen worden. In deze schifting zullen verschillende requirements opgezet worden waaraan de verschillende oplossingen zullen afgetoetst worden.
Uit deze schifting worden dan de beste twee kandidaten verkozen %waarin verder zal verdiept worden.

Na de schifting en de verdieping van de twee High Available PostgreSQL cluster oplossingen zal er in dit onderzoek één oplossing gebruikt worden waar een proof of concept mee gemaakt zal worden.

2. Opstelling schifting en keuze 2 oplossingen
% Bij de schifting is het belangrijk dat de oplossing open source is. Dat het een hedendaagse community heeft ( 2020 state-of-the-art) en de 3 punten waar Ruben waarden aan hechtte.


3. Vergelijkende studie van 2 oplossingen waarin ik deze 2 oplossing volledig uitleg en waarom deze juist gekozen zijn geweest.
% Hierin ga ik dan deze 2 oplossingen volledig toelichten en zeggen hoe zij juist in postgres oplossingen aanbieden op verschillende vlakken.


4. Keuze van 1 oplossing die ik zal voorleggen als ultieme oplossing, deze oplossing zal ook gebruikt worden voor de proof-of-concept
%Keuze oplossing uitleggen.

5. Proof-of-concept
%Aan de hand van foto's en uitleg de configuratie en installatie van de cluster omgeving uitleggen en dan van elke server zijn functionaliteiten uitleggen, en hoe deze in verbinding staat met de andere servers.



%%=============================================================================
%% Schifting 
%%=============================================================================

\chapter{\IfLanguageName{dutch}{Schifting}{Schifting}}
\label{ch:schifting}


\section{\IfLanguageName{dutch}{Requirements}{Requirements}}
\label{sec:Requirements}

\subsection{\IfLanguageName{dutch}{Functional Requirements}{Functional Requirements}}
\label{subsec:Functional Requirements}

\subsubsection{\IfLanguageName{dutch}{Redundancy}{Redundancy}}
\label{subsubsec:Redundancy}

\subsubsection{\IfLanguageName{dutch}{Failover}{Failover}}
\label{subsubsec:Failover}

\subsubsection{\IfLanguageName{dutch}{Monitoring}{Monitoring}}
\label{subsubsec:Monitoring}

\subsection{\IfLanguageName{dutch}{Non-functional Requirements}{Non-functional Requirements}}
\label{subsec:Non-functional Requirements}

\subsubsection{\IfLanguageName{dutch}{Open source}{Open source}}
\label{subsubsec:Open source}



\section{\IfLanguageName{dutch}{Hoe beantwoorden de verschillende oplossingen aan de bovenstaande schifting}{Hoe beantwoorden de verschillende oplossingen aan de bovenstaande schifting}}
\label{sec:Hoe beantwoorden de verschillende oplossingen aan de bovenstaande schifting}

% https://scalegrid.io/blog/managing-high-availability-in-postgresql-part-3/#:~:text=Patroni%20ensures%20the%20end%2Dto,be%20customized%20to%20your%20needs.

% https://www.slideshare.net/ScaleGrid/whats-the-best-postgresql-high-availability-framework-paf-vs-repmgr-vs-patroni-infographic

%In deze links worden 3 oplossingen vergeleken met elkaar





\subsection{\IfLanguageName{dutch}{Oplossing 1}{Oplossing 1}}
\label{subsec:Oplossing 1}

\subsection{\IfLanguageName{dutch}{Oplossing 2}{Oplossing 2}}
\label{subsec:Oplossing 2}

\subsection{\IfLanguageName{dutch}{Oplossing 3}{Oplossing 3}}
\label{subsec:Oplossing 3}

\subsection{\IfLanguageName{dutch}{Oplossing 4}{Oplossing 4}}
\label{subsec:Oplossing 4}

\subsection{\IfLanguageName{dutch}{Oplossing 5}{Oplossing 5}}
\label{subsec:Oplossing 5}

\subsection{\IfLanguageName{dutch}{Oplossing 6}{Oplossing 6}}
\label{subsec:Oplossing 6}
%%=============================================================================
%% Oplossing 1
%%=============================================================================

\chapter{Oplossing 1}

\label{ch:Oplossing 1}




%%=============================================================================
%% PostgreSQL Automatic Failover (PAF)
%%=============================================================================

\chapter{PostgreSQL Automatic Failover (PAF)}
\label{ch:PostgreSQL Automatic Failover (PAF)}

In dit hoofdstuk wordt meer informatie verschaft over PostgreSQL Automatic Failover (PAF) als een PostgreSQL High Availability cluster oplossing. Er zal worden ingegaan op de verschillende requirements, waaraan voldaan moet worden. Hierbij wordt dan telkens ook een score gegeven die in Hoofdstuk~\ref{ch:Verwerking resultaten}: Verwerking resultaten verder geanalyseerd zullen worden.

%https://dzone.com/articles/managing-high-availability-in-postgresql-part-i
%In dit hoofdstuk worden nog eens voor PostgreSQL Automatic Failover (PAF) alle punten uit de functionele requirements overlopen. Hier kunnen al meer specifieke commando's en tools benoemd worden die gebruikt kunnen worden voor de opstelling van een PostgreSQL Automatic Failover (PAF) cluster.
%https://wiki.postgresql.org/images/0/07/Ha_postgres.pdf

%Inleiding

\section{\IfLanguageName{dutch}{Inleiding tot PostgreSQL Automatic Failover (PAF)}{Inleiding tot PostgreSQL Automatic Failover (PAF)}}
\label{sec:Inleiding tot PostgreSQL Automatic Failover (PAF)}

PostgreSQL Automatic Failover (PAF) is een High Availability oplossing ontwikkeld door ClusterLabs die zich bezighoudt met het beheer van High Availability in PostgreSQL clusters. Het maakt vooral gebruik van Postgres synchrone replicatie om te garanderen dat er geen gegevens verloren gaan op het moment van de failover operatie. 

PostgreSQL Automatic Failover (PAF) werkt nauw samen met de tool Pacemaker. PAF is in staat om aan Pacemaker te laten zien wat de huidige status is van een node. Bij het optreden van een storing zal Pacemaker automatisch proberen dit te herstellen.
Als de storing niet te herstellen valt, dan zal PAF zorgen voor automatische failover

Pacemaker is een service die in staat is om veel resources te beheren, en doet dit met de hulp van hun resource agents. Resource agents hebben dan de verantwoordelijkheid om een specifieke resource af te handelen, hoe ze zich moeten gedragen, en Pacemaker te informeren over deze resultaten.

%Bronvermelding nog bijzetten

\section{\IfLanguageName{dutch}{Requirements}{Requirements}}
\label{sec:Requirements}

\subsection{\IfLanguageName{dutch}{Must have}{Must have}}
\label{subsec:Must have}

%PostgreSQL Automatic Failover (PAF) voldoet aan de nodige requirements, zeker met de tool Pacemaker is er heel veel mogelijk qua monitoring, failover en replicatie. De score hierbij voor “Must have” is 13/15 punten.


\subsubsection{\IfLanguageName{dutch}{Replicatie}{Replicatie}}
\label{subsubsec:Replicatie}

PostgreSQL Automatic Failover (PAF) zal niet 100\% kunnen beschermen tegen gegevensverlies. De replicatie wordt geconfigureerd met PostgreSQL. Bij asynchrone replicatie, wat de standaard is voor PostgreSQL, zullen de transacties eerst op primary node gecommitteerd worden voordat ze op de standby nodes zullen worden toegepast. In het geval van failover, zal de meest up-to-date standby node gepromoot worden door PostgreSQL Automatic Failover (PAF). Hierdoor zal gegevensverlies geminimaliseerd worden, maar er zal nog iets van verlies zijn. 

Voornamelijk zal PostgreSQL Automatic Failover (PAF) gebruik maken van synchrone replicatie om te garanderen dat er geen data verloren gaat tijdens een failover.

Hierdoor krijgt PostgreSQL Automatic Failover (PAF) voor deze requirement  een score van 4.

\subsubsection{\IfLanguageName{dutch}{Failover}{Failover}}
\label{subsubsec:Failover}

Aan de hand van Pacemaker kan er automatische failover plaatsvinden. PostgreSQL Automatic Failover (PAF) communiceert continu met Pacemaker over de status van de cluster en bewaakt het functioneren van de database. In geval van storing, wordt Pacemaker hierover geïnformeerd en zal het kijken om de storing op te lossen. Is de storing onoplosbaar, dan zal Pacemaker een verkiezing starten tussen de standby nodes om een opvolger van de primary node te selecteren.
Bij het configureren van de cluster wordt elke node geconfigureerd als standby node voor Pacemaker een keuze maakt van welke node de primay wordt. Dit zorgt er ook voor dat elke node weet hoe hij moet functioneren als een standby node.

'master-max' krijgt bij de configuratie van Pacemaker het aantal PostgreSQL nodes dat als primary zullen dienen op een gegeven moment. Default staat deze ingesteld op 1.

Bij 'clone-max' staat de ingestelde parameter gelijk aan het aantal nodes dat PostgreSQL kunnen draaien, primary of standby. Deze parameter staat normaal gelijk aan het aantal nodes die aanwezig zijn in de cluster.

'notify=true' laat toe om te signaleren wanneer er bepaalde acties gebeuren op de node. Dit zal in contact staan met Pacemaker. Deze staat standaard op inactief, dus moet de parameter liefst expliciet nog eens vermeld worden op 'true'.

PAF maakt gebruik van ip-adres failover in plaats van het herstarten van de standby node om verbinding te maken met de nieuwe master tijdens een failover event, wat voordelig is in scenario's waar een gebruiker de standby nodes niet wil herstarten.

Door de hierboven genoemde redenen krijgt PostgreSQL Automatic Failover (PAF) een score van 4 voor deze requirement.

\subsubsection{\IfLanguageName{dutch}{Monitoring}{Monitoring}}
\label{subsubsec:Monitoring}

Via de tool Pacemaker kan er bij PostgreSQL Automatic Failover (PAF) goed aan monitoring gedaan. Het zal signaleren wanneer een node niet bereikbaar is.
Het toevoegen van Watchdog als tool is hier mogelijk om ook aan monitoring te doen.

Hierdoor krijgt PostgreSQL Automatic Failover (PAF) voor monitoring een score van 4.


\subsection{\IfLanguageName{dutch}{Should have}{Should have}}
\label{subsec:Should have}


%Doordat er (nog) geen nieuwe release is in 2021, krijgt PostgreSQL Automatic Failover (PAF) ook 4/5 punten voor “Should have”, en geen 5/5.

\subsubsection{\IfLanguageName{dutch}{Actieve ondersteuning in 2020-2021}{Actieve ondersteuning in 2020-2021}}
\label{subsubsec:Actieve ondersteuning in 2020-2021}

De laatste release was op 10 maart 2020. Hieruit kunnen we concluderen dat ook deze oplossing nog up-to-date is en bruikbaar voor dit onderzoek. 

Hier krijgt PostgreSQL Automatic Failover (PAF) een score van 4. Een score van 5 ging behaald worden moest de laatste release in 2021 zijn.

\subsubsection{\IfLanguageName{dutch}{Open source}{Open source}}
\label{subsubsec:Open source}

PostgreSQL Automatic Failover (PAF) is een open source High Availability oplossing voor PostgreSQL clusters.

Voor deze requirement krijgt PostgreSQL Automatic Failover (PAF) een score van 5.

\subsection{\IfLanguageName{dutch}{Could have}{Could have}}
\label{subsec:Could have}

\subsubsection{\IfLanguageName{dutch}{Grafische interface}{Grafische interface}}
\label{subsubsec:Grafische interface}

Op eerste zicht is er geen algemeen gekende grafische interface aanwezig te zijn voor PostgreSQL Automatic Failover (PAF).

\subsubsection{\IfLanguageName{dutch}{Beperkte manuele interventie}{Beperkte manuele interventie}}
\label{subsubsec:Beperkte manuele interventie}

PostgreSQL Automatic Failover (PAF) kan zo geconfigureerd worden dat het weinig tot geen manuele interventie vereist. Aan de hand van bijvoorbeld Pacemaker is dit mogelijk.



\begin{table}[!h]
    \centering
    \begin{tabular}{ |p{6cm}||p{6cm}|  }
        \hline
        \multicolumn{2}{|c|}{Overzicht score PostgreSQL Automatic Failover (PAF)} \\
        \hline
        Must have & \\
        \hline
        Ondersteuning van replicatie  & 4 \\
        Ondersteuning van failover &  4 \\
        Ondersteuning van monitoring & 4 \\
        \hline
        Should have & \\
        \hline
        Open Source &  5 \\
        Ondersteuning in 2020-2021 & 4 \\
        \hline
        \hline
        Totaal & 21/25 \\
        \hline    
    \end{tabular}
    \caption{Overzicht score PostgreSQL Automatic Failover (PAF)}
    \label{table:Overzicht score PostgreSQL Automatic Failover (PAF)}
\end{table}

% Voeg hier je eigen hoofdstukken toe die de ``corpus'' van je bachelorproef
% vormen. De structuur en titels hangen af van je eigen onderzoek. Je kan bv.
% elke fase in je onderzoek in een apart hoofdstuk bespreken.

%\input{...}
%\input{...}
%...

%%=============================================================================
%% Conclusie
%%=============================================================================

\chapter{Conclusie}
\label{ch:conclusie}

In dit onderzoek wordt een antwoord gegeven op de onderzoeksvraag: “Welke PostgreSQL High Availability cluster oplossing kunnen bedrijven, de dag van vandaag, gebruiken om garantie te hebben op monitoring, redundantie en failover?” Om hierop een antwoord te vinden, is een vergelijkende studie uitgevoerd tussen de verschillende oplossingen uit dit onderzoek. Hierbij werd elke oplossing afgetoetst aan de requirements.

Uit de resultaten van de requirementsanalyse blijkt dat Patroni momenteel de PostgreSQL High Availability cluster oplossing is, die het meest aansluit aan de vooropgestelde requirements. De proof of concept beaamt ook deze vaststelling. Het gebruik van Patroni geeft een garantie op High Availability in een PostgreSQL cluster. Patroni zelf is op het vlak van replicatie zeer flexibel. Het werkt standaard met asynchrone replicatie, maar kan ook zo geconfigureerd worden dat het synchroon gebeurt. Wat betreft failover is Patroni een oplossing met een uitgebreide configuratie. Er zijn veel geavanceerde opties voor het configureren van failover in de cluster. Een handig iets is dat failover automatisch kan gebeuren en dus geen nood hoeft te hebben aan manuele interventies. Inzake monitoring kent Patroni een zeer rijke REST API die kan gebruikt worden om de cluster en de nodes te monitoren. Hierin kunnen persoonlijke voorkeuren van monitoring toegevoegd worden.

Ook PostgreSQL Automatic Failover (PAF) is een goed alternatief bij het opzetten van een PostgreSQL High Availability cluster. Het voldoet voldoende aan de verschillende (functionele) requirements om als alternatief gezien te worden van Patroni als oplossing. Aan de hand van de tool Pacemaker kan PostgreSQL Automatic Failover (PAF) zodanig geconfigureerd worden dat er failover en monitoring gedaan kan worden.

%De andere tools omschrijven waarom zij niet tot winnaar zijn gekomen en 
De overige twee tools: Pgpool-II en Replication Manager (repmgr) zijn zeker ook waardevolle oplossingen bij het opzetten van PostgreSQL High Availability cluster. Echter voldoen deze twee oplossingen minder aan Patroni en PostgreSQL Automatic Failover (PAF). 

Bij de aanvang van dit onderzoek had ik deze resultaten nog niet verwacht. Dit omdat de verschillende oplossingen nog onvoldoende aan bod waren gekomen om de resultaten te voorspellen. Het was pas na het lezen van het artikel van Akhtar, H: “PostgreSQL High Availability: The Considerations and Candidates” dat ik een idee kreeg van welke PostgreSQL oplossingen er allemaal aanwezig waren. Akhtar haalde Patroni, PostgreSQL Automatic Failover (PAF), pgPool-II en Replication Manager (repmgr) aan als gepaste oplossingen voor High Availability in een PostgreSQL omgeving. Uit de literatuurstudie bleek dat Patroni een erg sterke PostgreSQL High Availability oplossing bood. De requirementanalyse bevestigde en beargumenteerde dit. 
%meerwaarde
Dit onderzoek kan een basis vormen voor bedrijven die zich willen inwerken in PostgreSQL High Availability. Er wordt hier een duidelijk beeld gegeven van de verschillende oplossingen en de manier waarop deze oplossingen voldoen aan bepaalde requirements. Deze requirements zijn gekomen uit de bedrijfswereld en voldoen hierbij aan de noden van andere bedrijven. 

%vervolgonderzoek
Dit onderzoek vraagt zeker naar een vervolgonderzoek voor de toekomst. Hoe wordt High Availability binnen tien jaar geïmplementeerd in PostgreSQL clusters? Welke nieuwe noden in een cluster zijn er dan van belang?
Vervolgonderzoek zal zeker een meerwaarde bieden aan bedrijven die belang hechten aan de duurzaamheid van hun clusters.


%1. Antwoord op Hoofdonderzoeksvraag

%2. Antwoord op deelonderzoekvragen

% TODO: Trek een duidelijke conclusie, in de vorm van een antwoord op de
% onderzoeksvra(a)g(en). Wat was jouw bijdrage aan het onderzoeksdomein en
% hoe biedt dit meerwaarde aan het vakgebied/doelgroep? 
% Reflecteer kritisch over het resultaat. In Engelse teksten wordt deze sectie
% ``Discussion'' genoemd. Had je deze uitkomst verwacht? Zijn er zaken die nog
% niet duidelijk zijn?
% Heeft het onderzoek geleid tot nieuwe vragen die uitnodigen tot verder 
%onderzoek?




%%=============================================================================
%% Bijlagen
%%=============================================================================

\appendix
\renewcommand{\chaptername}{Appendix}

%%---------- Onderzoeksvoorstel -----------------------------------------------

\chapter{Onderzoeksvoorstel}

Het onderwerp van deze bachelorproef is gebaseerd op een onderzoeksvoorstel dat vooraf werd beoordeeld door de promotor. Dat voorstel is opgenomen in deze bijlage.

% Verwijzing naar het bestand met de inhoud van het onderzoeksvoorstel
%---------- Inleiding ---------------------------------------------------------

\section{Introductie} % The \section*{} command stops section numbering
\label{sec:introductie}
Database high availability (HA) staat voor de garantie van het behouden van gegevens in geval er zich een defect of storing voordoet aan de databank  server. Een storing aan een databank server kan te wijten zijn aan verschillende factoren. Voorbeelden hiervan zijn het verlies van netwerkconnectie en een defect in de software of hardware van de databankserver. Ook menselijke factoren en omgevingsfactoren moeten in rekening genomen worden. Voorbeelden hiervan zijn een menselijke vergissing en een wijziging in temperatuur. Investeren in high availability geeft meer zekerheid over de beschikbaarheid van data en biedt verschillende mogelijkheden voor failover en systeembescherming \autocite{IBM1}. Met behulp van clusters kan er één actieve en een of meerdere standby instanties van de databank server zijn. Een cluster is een groep van servers en computers die samenwerken met elkaar alsof het één systeem is. Deze standby instanties zullen, in het ideale geval, dezelfde gegevens bevatten als de actieve server~\autocite{BDQ}. Wanneer dan een actieve server faalt, kan een standby instantie inspringen waardoor dataverlies en server downtime gereduceerd worden.

Als proof-of-concept wordt er in dit onderzoek een PostgresSQL (pgSQL) cluster opgezet. PostgreSQL is een open-source, object-relationeel databank systeem~\autocite{PostgreSQL2020}. Bij Inuits, een Belgisch open-source bedrijf met verschillende vestigingen in Europa, merken ze een stijging in de vraag naar het PostgreSQL verhaal. Deze cluster wordt volledig geautomatiseerd en zal reproduceerbaar zijn, dit aan de hand van Puppet, een configuration management tool. Aan de hand van deze cluster zal dan getoond worden hoe database high availability (HA) in werking treedt.


%\begin{itemize}
%  \item de probleemstelling en context
%  \item de motivatie en relevantie voor het onderzoek
%  \item de doelstelling en onderzoeksvraag/-vragen
%\end{itemize}

%---------- Stand van zaken ---------------------------------------------------

\section{State of the art}
\label{sec:state of the art}
Over database high availability (HA) is er veel informatie te vinden. Een Google search naar "database high availability" levert in minder dan één seconde 733.000.000 resultaten op. Wanneer ik hier "open source" aan toevoeg, zijn er nog steeds 433.000.000 resultaten beschikbaar. Over database high availability zijn er dus veel artikels beschikbaar die nuttig kunnen zijn voor dit onderzoek. Er zijn ook heel veel fora die antwoorden bieden op vragen van personen omtrent database high availability (HA), open-source en SQL servers.


Over high availability (HA) zijn er voldoende artikels, blogs... Eén artikel spreekt over high availability clustering en hoe dit kan aangepakt worden. Het bespreekt de cluster architectuur en wat de best practice is voor high availability binnen een cluster. De conclusie die hier getrokken wordt is dat het primaire doel van een high availability (HA) systeem is om te voorkomen en elimineren van alle single points of failure. Deze moeten beschikken over meerdere geteste actieplannen. Dit zodat ze in geval van storing, verstoring en defect in dienstverlening direct, gepast en onafhankelijk kunnen reageren. 
Zorgvuldige planning + betrouwbare implementatiemethoden + stabiele softwareplatforms + degelijke hardware-infrastructuur + vlotte technische operaties + voorzichtige managementdoelstellingen + consistente databeveiliging + voorspelbare redundantiesystemen + robuuste back-upoplossingen + meerdere herstelopties = 100\% uptime \autocite{Singer2020}.
In veel artikels zien we meer of mindere punten, maar het zijn wel vaak dezelfde die altijd terugkomen. In een ander artikel wordt een highly available (HA) infrastructuur gekenmerkt door: 1. Hardware redundantie; 2. Software en applicatie redundantie; 3. Gegevens redundantie; 4. Elimineren van storingspunten \autocite{Jevtic2018}.


Er is een artikel die vier van de meest gebruikte database high availability (HA) tools/oplossingen oplijst en deze ook uitlegt. Deze vier zijn "PgPool-II", "PostgreSQL Automatic Failover (PAF)", "RepMgr [Replication Manager] ", en "Patroni" \autocite{Akhtar2020}. Dit artikel is zeker de moeite waard om door te nemen omdat in dit artikel eigenschappen van deze vier oplossingen vergeleken worden met elkaar. De website zelf biedt ook veel links aan naar gerelateerde artikels over PostgreSQL en high availability (HA). 
MySQL zelf toont op hun website verschillende manieren van hoe zij high availability implementeren in een cluster \autocite{MySQL2021}. Deze info zal nuttig zijn voor in dit onderzoek. Via MySQL zal ik vergelijkingen kunnen maken en zal ik eventueel lijnen kunnen doortrekken naar PostgreSQL.



De top drie open-source databanken van 2019 zijn, in volgorde van top 3, MySQL met 31.7\%, PostgreSQL met 13.4\% en MongoDB met 12.2\% \autocite{Anderson2020} van het totaal aantal open-source databank gebruikers.


Over het opzetten van een PostgreSQL (pgSQL) cluster zijn er ook veel artikels te vinden. In een bepaalde blog wordt er een high available (HA) PostgreSQL cluster \autocite{2019} opgezet aan de hand van Patroni op een Ubuntu Linux-distributie. Deze blog zal zeer interessant zijn voor dit onderzoek omdat er ook in getest wordt. Dit zal handig zijn om te kunnen vergelijken bij het opzetten en testen van de proof-of-concept .

%Hier beschrijf je de \emph{state-of-the-art} rondom je gekozen onderzoeksdomein. Dit kan bijvoorbeeld een literatuurstudie zijn. Je mag de titel van deze sectie ook aanpassen (literatuurstudie, stand van zaken, enz.). Zijn er al gelijkaardige onderzoeken gevoerd? Wat concluderen ze? Wat is het verschil met jouw onderzoek? Wat is de relevantie met jouw onderzoek?

%Verwijs bij elke introductie van een term of bewering over het domein naar de vakliteratuur, bijvoorbeeld~\autocite{Doll1954}! Denk zeker goed na welke werken je refereert en waarom.~\autocite{Smits2011}

% Voor literatuurverwijzingen zijn er twee belangrijke commando's:
% \autocite{KEY} => (Auteur, jaartal) Gebruik dit als de naam van de auteur
%   geen onderdeel is van de zin.
% \textcite{KEY} => Auteur (jaartal)  Gebruik dit als de auteursnaam wel een
%   functie heeft in de zin (bv. ``Uit onderzoek door Doll & Hill (1954) bleek
%   ...'')

%Je mag gerust gebruik maken van subsecties in dit onderdeel.

%---------- Methodologie ------------------------------------------------------
\section{Methodologie}
\label{sec:methodologie}
In de eerste fase van het onderzoek zal er een vergelijkende studie gebeuren over de huidige, anno 2020, database high availability (HA) tooling. Deze verschillende tools/oplossingen zullen dan met elkaar vergeleken worden. Hierbij wordt gekeken naar welke elementen er allemaal (meermaals) voorkomen. Hiervan komt er een lijst die gebruikt zal worden om te schiften tussen de verschillende oplossingen. Op deze manier zal er dan een oplossing gekozen worden.  In de tweede fase van het onderzoek wordt de focus gelegd op het opzetten van de PostgreSQL (pgSQL) cluster. Deze zal vooraf gegaan worden door een voorbereidende studie over PostgreSQL (pgSQL). Aan de hand hiervan zal er gewerkt worden aan het opbouwen van de PostgreSQL (pgSQL) cluster. Vooraleer dit geautomatiseerd wordt, zal de opbouw manueel verlopen. Wanneer deze manueel een succes is, zal er gewerkt worden aan de automatisatie van de PostgreSQL (pgSQL) cluster. De opbouw zal gebeuren via virtuele machines (VirtualBox) waarop Linux-distributies staan. In het onderzoek wordt Ubuntu gebruikt. Deze keuze kan wijzigen naargelang het verloop van het onderzoek. De opbouw zal telkens grondig gedocumenteerd worden. Alle commando's zullen overlopen worden. Hierna zal er een inleidende studie zijn over Puppet. Hierna zal er via Puppet gewerkt worden om deze PostgreSQL (pgSQL) cluster te reproduceren. Ook hier zal alles grondig gedocumenteerd worden.
Na het opzetten van de PostgreSQL (pgSQL) cluster zullen er verschillende experimenten zijn die de high availability (HA) zullen testen.




%Hier beschrijf je hoe je van plan bent het onderzoek te voeren. Welke onderzoekstechniek ga je toepassen om elk van je onderzoeksvragen te beantwoorden? Gebruik je hiervoor experimenten, vragenlijsten, simulaties? Je beschrijft ook al welke tools je denkt hiervoor te gebruiken of te ontwikkelen.

%---------- Verwachte resultaten ----------------------------------------------
\section{Verwachte resultaten}
\label{sec:verwachte_resultaten}
Uit het onderzoek zal duidelijk blijken dat verschillende database high availability (HA) tools mogelijk zijn binnen een PostgreSQL (pgSQL) cluster. De opbouw van deze cluster zal gebeuren aan de hand van virtuele machines. En wanneer de virtuele machine, waarop de PostgreSQL server staat, uitvalt, zal er een standby instantie van deze server het werk van de actieve, uitgevallen server overnemen. Hierdoor zal er geen downtime of dataverlies zijn. De data zal beschikbaar en onverstoord blijven.

Uit dit onderzoek zullen ook best practices volgen die gehanteerd kunnen worden om op die manier het risico op verlies van gegevens te verminderen. De kans om offline te zijn zal lager liggen met een high available systeem.

Door gebruik te maken van Puppet zal de reproduceerbaarheid van de high available (HA) PostgreSQL (pgSQL) cluster zeer eenvoudig en snel zijn.
%Via Puppet op een snelle, moeiteloze manier een high available (HA) PostgreSQL (pgSQL) cluster kunnen opzetten en configureren. Dit zal kunnen doordat de cluster geautomatiseerd en reproduceerbaar zal zijn.


%Hier beschrijf je welke resultaten je verwacht. Als je metingen en simulaties uitvoert, kan je hier al mock-ups maken van de grafieken samen met de verwachte conclusies. Benoem zeker al je assen en de stukken van de grafiek die je gaat gebruiken. Dit zorgt ervoor dat je concreet weet hoe je je data gaat moeten structureren.

%---------- Verwachte conclusies ----------------------------------------------
\section{Verwachte conclusies}
\label{sec:verwachte_conclusies}
Uit het onderzoek zal blijken dat database high availability (HA) een blijvend topic is waar voldoende aandacht aan besteedt moet worden. In kleine bedrijven hoeft high availability (HA) geen al te grote prioriteit te hebben, maar naarmate een bedrijf groeit, zal high availability (HA) steeds belangrijker worden. Zonder de implementatie van een high available (HA) architectuur kan een storing of downtime van de databank (SQL) server  grote gevolgen hebben op een bedrijf/organisatie. Gevolgen zoals verlies van vertrouwen bij klanten, verlies van inkomen, verlies van informatie. Door middel van hardware-, software- en gegevensredundantie en het elimineren van mogelijke storingspunten gaan we high availability kunnen blijven garanderen.
De kost en tijd die gïnvesteerd moet worden in het onderhouden van een high available systeem zal lager liggen dan de kost en tijd die geïnvesteerd moet worden in geval van een downtime. 

Uit dit onderzoek zal ook blijken dat PostgreSQL (pgSQL) een volwaardig alternatief is van MySQL bij het opzetten van SQL clusters.

%Hier beschrijf je wat je verwacht uit je onderzoek, met de motivatie waarom. Het is \textbf{niet} erg indien uit je onderzoek andere resultaten en conclusies vloeien dan dat je hier beschrijft: het is dan juist interessant om te onderzoeken waarom jouw hypothesen niet overeenkomen met de resultaten.


%%---------- Andere bijlagen --------------------------------------------------
% TODO: Voeg hier eventuele andere bijlagen toe
%\input{...}

%%---------- Referentielijst --------------------------------------------------

\printbibliography[heading=bibintoc]

\end{document}
