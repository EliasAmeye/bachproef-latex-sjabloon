%%=============================================================================
%% Oplossing 1
%%=============================================================================

\chapter{Oplossing 1: Patroni}
\label{ch:Oplossing 1: Patroni}

In dit hoofdstuk worden nog eens alle punten uit de functionele requirements uitgebreid overlopen.

NOG AANVULLEN!!!

%Nog eens volledige uitleg en alle punten en tools zeer grondig doorlopen en vergelijken met oplossing 2

\section{\IfLanguageName{dutch}{Redundantie/Replicatie}{Redundantie/Replicatie}}
\label{sec:Redundantie/Replicatie}

Voor het kopiëren van bestanden of databases, of replicatie genaamd maakt Patroni gebruikt van streaming replicatie. Standaard maakt Patroni gebruik van asynchrone replicatie. Hierbij worden gegevens eerst geschreven naar een primaire opslagarray en committeert dan de te repliceren gegevens naar het geheugen. Vervolgens worden deze gegevens in real-time of met geplande tussenpozen naar de replicatiedoelen gekopieerd.
Wanneer gewerkt wordt met asynchrone replicatie, kan de cluster het zich permitteren om een aantal gecommitteerde transacties te verliezen, om High Availability te garanderen. Wanneer de primary node faalt, en Patroni een standby node regelt, zullen alle transacties die niet naar de standby node zijn gerepliceerd, verloren gaan. Het aantal transacties dat verloren mag gaan, kan geregeld worden aan de hand van “maximum\_lag\_on\_failover”. Dit zal bij failover het aantal verloren transacties beperken.

PostgreSQL synchrone replicatie is ook mogelijk bij Patroni. Synchrone replicatie zorgt voor meer consistentie van data in een cluster door te bevestigen dat schrijfacties naar een secundary node worden geschreven voordat ze naar de verbindende client terugkeren met een succes. De nadelen van synchrone replicatie houden in dat er een verminderde verwerkingscapaciteit is voor schrijfacties. Deze verwerkingscapaciteit is volledig gebaseerd op netwerkprestaties. Het gebruik van PostgreSQL synchrone replicatie garandeert niet dat er onder alle omstandigheden geen transacties verloren zullen gaan. Wanneer de primary node en de secundary node die op dat moment synchrone replicatie draaien, gelijktijdig falen, zal een derde node, die mogelijk niet alle transacties bevat, worden gepromoveerd tot primary node.

% https://patroni.readthedocs.io/en/latest/replication_modes.html


\section{\IfLanguageName{dutch}{Failover}{Failover}}
\label{sec:Failover}

\section{\IfLanguageName{dutch}{Monitoring}{Monitoring}}
\label{sec:Monitoring}




%\subsection{\IfLanguageName{dutch}{Databank}{Databank}}
%\label{subsec:Databank}



