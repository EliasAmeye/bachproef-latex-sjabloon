%---------- Inleiding ---------------------------------------------------------
\section{Samenvatting}
\label{sec:samenvatting}
\Abstract{
    In deze bachelorproef zal onderzoek gedaan worden naar de huidige staat van open-source database high availability (HA) tooling. De vraag naar PostgreSQL (pgSQL) wordt steeds groter, waardoor een onderzoek naar open-source database high availability (HA) tooling zeker niet onmisbaar is.
    In het eerste deel van dit onderzoek zal er geduid worden wat open-source database high availability (HA) tooling precies is en wat de toepassingen ervan zijn. Hierna wordt een vergelijkende studie uitgevoerd over de verschillende open-source database high availability (HA) oplossingen. %Uit deze studie en in samenwerking met Inuits zal gekeken worden welke open-source database HA oplossing/tool er kunnen uitgerold worden binnen de infrastructuur van Inuits, om zo ook de komende 10 jaar aan de vraag van database HA te kunnen voldoen.
    Na de vergelijkende studie wordt als proof-of-concept een PostgreSQL (pgSQL) cluster opgezet. %met behulp van de automation tool Puppet. Dit zorgt voor automatisering en reproduceerbaarheid.
    In dit onderzoek zal ik worden bijgestaan door Ruben Demey, Global IT Operations Manager bij ST Engineering iDirect. Het verwachte resultaat van dit onderzoek is aantonen dat er een PostgreSQL high availability (ha) cluster oplossing aanwezig is die voldoet aan de requirements, opgesteld in dit onderzoek. In de toekomst is het interessant om soortgelijk onderzoek uit te voeren, om te kijken of de tools die in dit onderzoek worden aangehaald, nog steeds even waardevol zullen zijn.
}


\section{Introductie} % The \section*{} command stops section numbering
\label{sec:introductie}
Database high availability (HA) staat voor de garantie van het behouden van gegevens in geval er zich een defect of storing voordoet aan de databank  server. Een storing aan een databank server kan te wijten zijn aan verschillende factoren. Voorbeelden hiervan zijn het verlies van netwerkconnectie en een defect in de software of hardware van de databankserver. Ook menselijke factoren en omgevingsfactoren moeten in rekening genomen worden. Voorbeelden hiervan zijn een menselijke vergissing en een wijziging in temperatuur. Investeren in high availability geeft meer zekerheid over de beschikbaarheid van data en biedt verschillende mogelijkheden voor failover en systeembescherming \autocite{IBM1}. Met behulp van clusters kan er één actieve en een of meerdere standby instanties van de databank server zijn. Een cluster is een groep van servers en computers die samenwerken met elkaar alsof het één systeem is. Deze standby instanties zullen, in het ideale geval, dezelfde gegevens bevatten als de actieve server~\autocite{BDQ}. Wanneer dan een actieve server faalt, kan een standby instantie inspringen waardoor dataverlies en server downtime gereduceerd worden.

%Als proof-of-concept wordt er in dit onderzoek een PostgresSQL (pgSQL) cluster opgezet. PostgreSQL is een open-source, object-relationeel databank systeem~\autocite{PostgreSQL2020}. Bij Inuits, een Belgisch open-source bedrijf met verschillende vestigingen in Europa, merken ze een stijging in de vraag naar het PostgreSQL verhaal. Deze cluster wordt volledig geautomatiseerd en zal reproduceerbaar zijn, dit aan de hand van Puppet, een configuration management tool. Aan de hand van deze cluster zal dan getoond worden hoe database high availability (HA) in werking treedt.


%\begin{itemize}
%  \item de probleemstelling en context
%  \item de motivatie en relevantie voor het onderzoek
%  \item de doelstelling en onderzoeksvraag/-vragen
%\end{itemize}

%---------- Stand van zaken ---------------------------------------------------

\section{State of the art}
\label{sec:state of the art}
%Over database high availability (HA) is er veel informatie te vinden. Een Google search naar "database high availability" levert in minder dan één seconde 733.000.000 resultaten op. Wanneer ik hier "open source" aan toevoeg, zijn er nog steeds 433.000.000 resultaten beschikbaar. Over database high availability zijn er dus veel artikels beschikbaar die nuttig kunnen zijn voor dit onderzoek. Er zijn ook heel veel fora die antwoorden bieden op vragen van personen omtrent database high availability (HA), open-source en SQL servers.


%Over high availability (HA) zijn er voldoende artikels, blogs... 
Singer spreekt over high availability (HA) clustering als een groep van servers die applicaties en services ondersteunen die op een betrouwbare manier gebruikt kunnen worden met een minimaal aantal aantal downtimes. Hij bespreekt de cluster architectuur en wat de best practice is voor high availability (HA) binnen een cluster. De conclusie die hier getrokken wordt, is dat het primaire doel van een high availability (HA) systeem het voorkomen en elimineren van alle single points of failure zijn. Dit systeem moet beschikken over meerdere geteste actieplannen. Dit zodat ze in geval van storing, verstoring en defect in dienstverlening direct, gepast en onafhankelijk kunnen reageren. 
Zorgvuldige planning + betrouwbare implementatiemethoden + stabiele softwareplatforms + degelijke hardware-infrastructuur + vlotte technische operaties + voorzichtige managementdoelstellingen + consistente databeveiliging + voorspelbare redundantiesystemen + robuuste back-upoplossingen + meerdere herstelopties = 100\% uptime \autocite{Singer2020}.
Ook Jevtic heeft het over veel van de punten die hierboven reeds zijn aangehaald. Jevtic spreekt over een highly available architectuur waarin meerdere componenten samenwerken om een ononderbroken service gedurende een bepaalde periode te garanderen. Dit omvat ook de reactietijd op verzoeken van gebruikers. Jevtic kenmerkt een  highly available (HA) infrastructuur aan de hand van: 1. Hardware redundantie; 2. Software en applicatie redundantie; 3. Gegevens redundantie; 4. Elimineren van storingspunten \autocite{Jevtic2018}.
Akhtar heeft vier van de meest gebruikte database high availability (HA) oplossingen opgelijst. Deze vier zijn "PgPool-II", "PostgreSQL Automatic Failover (PAF)", "RepMgr [Replication Manager] ", en "Patroni" \autocite{Akhtar2020}. Akhtar vergelijkt deze verschillende oplossingen kort met elkaar. Akhtar definieert high availability (HA) als niet alleen de continuïteit van een bepaalde service, maar volgens hem gaat high availability (HA) ook over het vermogen van een systeem om een (hogere) werkdruk te kunnen schalen en te beheren. Dit systeem moet volgens Akhtar de gemiddelde werkdruk, maar ook de piekmomenten aankunnen.
%Dit artikel is zeker de moeite waard om door te nemen omdat in dit artikel eigenschappen van deze vier oplossingen vergeleken worden met elkaar. De website zelf biedt ook veel links aan naar gerelateerde artikels over PostgreSQL en high availability (HA). 
%MySQL zelf toont op hun website verschillende manieren van hoe zij high availability implementeren in een cluster \autocite{MySQL2021}. Deze info zal nuttig zijn voor in dit onderzoek. Via MySQL zal ik vergelijkingen kunnen maken en zal ik eventueel lijnen kunnen doortrekken naar PostgreSQL.
Aldus Andersen zijn de top drie open-source databanken van 2019, in volgorde van top 3, MySQL met 31.7\%, PostgreSQL met 13.4\% en MongoDB met 12.2\% \autocite{Anderson2020} van het totaal aantal open-source databank gebruikers.


%Over het opzetten van een PostgreSQL (pgSQL) cluster zijn er ook veel artikels te vinden. In een bepaalde blog wordt er een high available (HA) PostgreSQL cluster \autocite{2019} opgezet aan de hand van Patroni op een Ubuntu Linux-distributie. Deze blog zal zeer interessant zijn voor dit onderzoek omdat er ook in getest wordt. Dit zal handig zijn om te kunnen vergelijken bij het opzetten en testen van de proof-of-concept .

%Hier beschrijf je de \emph{state-of-the-art} rondom je gekozen onderzoeksdomein. Dit kan bijvoorbeeld een literatuurstudie zijn. Je mag de titel van deze sectie ook aanpassen (literatuurstudie, stand van zaken, enz.). Zijn er al gelijkaardige onderzoeken gevoerd? Wat concluderen ze? Wat is het verschil met jouw onderzoek? Wat is de relevantie met jouw onderzoek?

%Verwijs bij elke introductie van een term of bewering over het domein naar de vakliteratuur, bijvoorbeeld~\autocite{Doll1954}! Denk zeker goed na welke werken je refereert en waarom.~\autocite{Smits2011}

% Voor literatuurverwijzingen zijn er twee belangrijke commando's:
% \autocite{KEY} => (Auteur, jaartal) Gebruik dit als de naam van de auteur
%   geen onderdeel is van de zin.
% \textcite{KEY} => Auteur (jaartal)  Gebruik dit als de auteursnaam wel een
%   functie heeft in de zin (bv. ``Uit onderzoek door Doll & Hill (1954) bleek
%   ...'')

%Je mag gerust gebruik maken van subsecties in dit onderdeel.

%---------- Methodologie ------------------------------------------------------
\section{Methodologie}
\label{sec:methodologie}
In de eerste fase van het onderzoek zal er een vergelijkende studie gebeuren over de huidige, database high availability (HA) oplossingen. Deze verschillende tools/oplossingen zullen dan met elkaar vergeleken worden. Hierbij wordt gekeken naar welke elementen er allemaal (meermaals) voorkomen. Hiervan komt er een lijst die gebruikt zal worden om te schiften tussen de verschillende oplossingen. Op deze manier zal er dan een oplossing gekozen worden. Via deze methode wordt er gekeken om maximum 3 verschillende oplossingen uit te werken, waarbij één oplossing gebruikt zal worden bij de proof of concept.
In de tweede fase van het onderzoek wordt de focus gelegd op het opzetten van de PostgreSQL (pgSQL) cluster als proof of concept. PostgreSQL is een open-source, object-relationeel databank systeem~\autocite{PostgreSQL2020}. Bij Inuits, een Belgisch open-source bedrijf met verschillende vestigingen in Europa, merken ze een stijging in de vraag naar het PostgreSQL verhaal.
Deze zal vooraf gegaan worden door een literatuurstudie over PostgreSQL (pgSQL). Aan de hand hiervan zal er gewerkt worden aan het opbouwen van de PostgreSQL (pgSQL) cluster. Vooraleer dit geautomatiseerd wordt, zal de opbouw manueel verlopen. De opbouw zal gebeuren via virtuele machines (VirtualBox) waarop Linux-distributies staan. In het onderzoek zal dus gebruik gemaakt worden van Linux-servers. De keuze van Linux-distributie zal onderbouwd worden in dit onderzoek. De opbouw van de cluster zal telkens grondig gedocumenteerd worden. %Alle commando's zullen hierbij overlopen worden. Hierna zal er een inleidende literatuurstudie zijn over Puppet en zal er aansluitend via Puppet gewerkt worden om deze PostgreSQL (pgSQL) cluster te reproduceren. Ook hier zal alles grondig gedocumenteerd worden.
Na het opzetten van de PostgreSQL (pgSQL) cluster zal er getest worden of de gebruikte database high availability (HA) oplossing functioneel is. Deze testen zullen uitgebreid beschreven worden.


%Als proof-of-concept wordt er in dit onderzoek een PostgresSQL (pgSQL) cluster opgezet. PostgreSQL is een open-source, object-relationeel databank systeem~\autocite{PostgreSQL2020}. Bij Inuits, een Belgisch open-source bedrijf met verschillende vestigingen in Europa, merken ze een stijging in de vraag naar het PostgreSQL verhaal. Deze cluster wordt volledig geautomatiseerd en zal reproduceerbaar zijn, dit aan de hand van Puppet, een configuration management tool. Aan de hand van deze cluster zal dan getoond worden hoe database high availability (HA) in werking treedt.


%Hier beschrijf je hoe je van plan bent het onderzoek te voeren. Welke onderzoekstechniek ga je toepassen om elk van je onderzoeksvragen te beantwoorden? Gebruik je hiervoor experimenten, vragenlijsten, simulaties? Je beschrijft ook al welke tools je denkt hiervoor te gebruiken of te ontwikkelen.

%---------- Verwachte resultaten ----------------------------------------------
\section{Verwachte resultaten}
\label{sec:verwachte_resultaten}
Uit het onderzoek zal blijken dat verschillende database high availability (HA) oplossingen mogelijk zijn binnen een PostgreSQL (pgSQL) cluster. De opbouw van deze cluster zal gebeuren aan de hand van virtuele machines. Wanneer de virtuele machine, waarop de PostgreSQL server staat, uitvalt, zal er een standby instantie van deze server het werk van de actieve, uitgevallen server overnemen. Hierdoor zal er geen downtime of dataverlies zijn. De data zal beschikbaar en onverstoord blijven.

Uit dit onderzoek zullen ook best practices volgen die gehanteerd kunnen worden om op die manier het risico op verlies van gegevens te verminderen. De kans om offline te zijn zal lager liggen met een high available systeem.

%Door gebruik te maken van Puppet zal de reproduceerbaarheid van de high available (HA) PostgreSQL (pgSQL) cluster zeer eenvoudig en snel moeten verlopen.
%Via Puppet op een snelle, moeiteloze manier een high available (HA) PostgreSQL (pgSQL) cluster kunnen opzetten en configureren. Dit zal kunnen doordat de cluster geautomatiseerd en reproduceerbaar zal zijn.


%Hier beschrijf je welke resultaten je verwacht. Als je metingen en simulaties uitvoert, kan je hier al mock-ups maken van de grafieken samen met de verwachte conclusies. Benoem zeker al je assen en de stukken van de grafiek die je gaat gebruiken. Dit zorgt ervoor dat je concreet weet hoe je je data gaat moeten structureren.

%---------- Verwachte conclusies ----------------------------------------------
\section{Verwachte conclusies}
\label{sec:verwachte_conclusies}
Uit het onderzoek zal blijken dat database high availability (HA) een blijvend topic is waar voldoende aandacht aan besteed moet worden in kleine en grote bedrijven. 
%High availability (HA) hoeft high availability (HA) geen al te grote prioriteit te hebben, maar naarmate een bedrijf groeit, zal high availability (HA) steeds belangrijker worden. 
Zonder de implementatie van een high available (HA) architectuur kan een storing of downtime van de databank (SQL) server  grote gevolgen hebben op een bedrijf/organisatie. Gevolgen zoals verlies van vertrouwen bij klanten, verlies van inkomen, verlies van informatie. Door middel van hardware-, software- en gegevensredundantie en het elimineren van mogelijke storingspunten zal er high availability gegarandeerd worden.
De kost en tijd die geïnvesteerd moet worden in het onderhouden van een high available systeem zal lager liggen dan de kost en tijd die geïnvesteerd moet worden in geval van een downtime. 
Met de proof of concept wordt dan aangetoond dat database high availability (HA) eenvoudig te implementeren valt in een PostgreSQL (pgSQL) cluster.


%Uit dit onderzoek zal ook blijken dat PostgreSQL (pgSQL) een volwaardig alternatief is van MySQL bij het opzetten van SQL clusters.
%Hier beschrijf je wat je verwacht uit je onderzoek, met de motivatie waarom. Het is \textbf{niet} erg indien uit je onderzoek andere resultaten en conclusies vloeien dan dat je hier beschrijft: het is dan juist interessant om te onderzoeken waarom jouw hypothesen niet overeenkomen met de resultaten.
