%\title{LaTeX Portrait Poster Template}
%%%%%%%%%%%%%%%%%%%%%%%%%%%%%%%%%%%%%%%%%
% a0poster Portrait Poster
% LaTeX Template
% Version 1.0 (22/06/13)
%
% The a0poster class was created by:
% Gerlinde Kettl and Matthias Weiser (tex@kettl.de)
% 
% Adapter by Jens Buysse for Hogeschool Gent
% This template has been downloaded from:
% http://www.LaTeXTemplates.com
%
% License:
% CC BY-NC-SA 3.0 (http://creativecommons.org/licenses/by-nc-sa/3.0/)
%
%%%%%%%%%%%%%%%%%%%%%%%%%%%%%%%%%%%%%%%%%

%----------------------------------------------------------------------------------------
%	PACKAGES AND OTHER DOCUMENT CONFIGURATIONS
%----------------------------------------------------------------------------------------

\documentclass[a0,portrait]{a0poster}

\usepackage{multicol} % This is so we can have multiple columns of text side-by-side
\columnsep=100pt % This is the amount of white space between the columns in the poster
\columnseprule=3pt % This is the thickness of the black line between the columns in the poster

\usepackage[svgnames]{xcolor} % Specify colors by their 'svgnames', for a full list of all colors available see here: http://www.latextemplates.com/svgnames-colors

\usepackage{times} % Use the times font
%\usepackage{palatino} % Uncomment to use the Palatino font

\usepackage{graphicx} % Required for including images
\graphicspath{{figures/}} % Location of the graphics files
\usepackage{booktabs} % Top and bottom rules for table
\usepackage[font=small,labelfont=bf]{caption} % Required for specifying captions to tables and figures
\usepackage{amsfonts, amsmath, amsthm, amssymb} % For math fonts, symbols and environments
\usepackage{wrapfig} % Allows wrapping text around tables and figures
\usepackage[export]{adjustbox}

\begin{document}

%----------------------------------------------------------------------------------------
%	POSTER HEADER 
%----------------------------------------------------------------------------------------

% The header is divided into two boxes:
% The first is 75% wide and houses the title, subtitle, names, university/organization and contact information
% The second is 25% wide and houses a logo for your university/organization or a photo of you
% The widths of these boxes can be easily edited to accommodate your content as you see fit

\begin{minipage}[t]{0.75\linewidth}
\VeryHuge \color{HoGentAccent1} \textbf{High Availability oplossingen voor PostgreSQL: een vergelijkende studie en proof of concept} \color{Black}\\ % Title
%\Huge\textit{Ondertitel (eventueel)}\\[2.4cm] % Subtitle
\huge \textbf{Ameye Elias, Demey Ruben, Aelbrecht Thomas}\\[0.5cm] % Author(s)
\huge Hogeschool Gent, Valentin Vaerwyckweg 1, 9000 Gent\\[0.4cm] % University/organization
\Large \texttt{elias.ameye@hogent.be} \\
\end{minipage}
%
\begin{minipage}[t]{0.25\linewidth}
\includegraphics[width=13cm,right]{figures/HOGENT_Logo_Pos_rgb.png} 

\end{minipage}

\vspace{1cm} % A bit of extra whitespace between the header and poster content

%----------------------------------------------------------------------------------------

\begin{multicols}{2} % This is how many columns your poster will be broken into, a portrait poster is generally split into 2 columns

%----------------------------------------------------------------------------------------
%	ABSTRACT
%----------------------------------------------------------------------------------------

\color{HoGentAccent1} % Navy color for the abstract

\begin{abstract}

\end{abstract}
%----------------------------------------------------------------------------------------
%	INTRODUCTION
%----------------------------------------------------------------------------------------

\color{HoGentAccent1} 
\section*{Introductie}
\color{black}
\color{black}
High Availability staat voor de garantie van het behouden van gegevens in geval er zich een defect of storing voordoet aan de database server. Een storing aan een databank server kan te wijten zijn aan verschillende factoren. Voorbeelden hiervan zijn het verlies van netwerkconnectie en een defect in de software of hardware van de databaseserver. Ook menselijke factoren en omgevingsfactoren moeten in rekening genomen worden. Voorbeelden hiervan zijn een menselijke vergissing en een wijziging in temperatuur. Investeren in High Availability geeft meer zekerheid over de beschikbaarheid van data en biedt verschillende mogelijkheden voor failover en systeembescherming. 
Met behulp van clustering kan er één actieve en een of meerdere standby nodes van de database server aanwezig zijn. Een cluster is een groep van servers en computers die samenwerken met elkaar alsof het één systeem is. Via deze manier zullen de standby nodes, in het ideale geval, dezelfde gegevens bevatten als de actieve node. Wanneer dan een actieve node faalt, kan een standby node inspringen waardoor dataverlies en server downtime gereduceerd worden.


%----------------------------------------------------------------------------------------
%	GEOLOGY
%----------------------------------------------------------------------------------------

\color{Black} % DarkSlateGray color for the rest of the content
\color{HoGentAccent1} 
\section*{Experimenten}
\color{black}



\color{HoGentAccent1} 
\section*{Sectie met figuur}
\color{black}


\begin{center}\vspace{1cm}
\includegraphics[width=1.0\linewidth]{grail}
\captionof{figure}{\color{HoGentAccent5} He hasn't got shit all over him. The nose? Where'd you get the coconuts? What do you mean? We shall say 'Ni' again to you, if you do not appease us}
\end{center}\vspace{1cm}

%------------------------------------------------



\color{HoGentAccent1} 
\section*{Conclusie}
\color{black}

Uit de resultaten van de requirementsanalyse bleek dat Patroni momenteel de PostgreSQL High Availability cluster oplossing is die het meest aansluit aan de vooropgestelde requirements. De proof of concept beaamt ook deze vaststelling. Het gebruik van Patroni geeft een garantie op High Availability in een PostgreSQL cluster. Patroni zelf is op het vlak van replicatie zeer flexibel. Het werkt standaard met asynchrone replicatie, maar kan ook zo geconfigureerd worden dat het synchroon gebeurt. Wat betreft failover is Patroni een oplossing met een uitgebreide configuratie. Er zijn veel geavanceerde opties voor het configureren van failover in de cluster. Een nuttige optie is dat failover automatisch kan gebeuren en dus geen nood hoeft te hebben aan manuele interventies. Inzake monitoring kent Patroni een zeer rijke REST API die kan gebruikt worden om de cluster en de nodes te monitoren. Hierin kunnen persoonlijke voorkeuren van monitoring toegevoegd worden.

Ook PostgreSQL Automatic Failover (PAF) is een goed alternatief bij het opzetten van een PostgreSQL High Availability cluster. Het voldoet voldoende aan de verschillende (functionele) requirements om als alternatief gezien te worden van Patroni als oplossing. Aan de hand van de tool Pacemaker kan PostgreSQL Automatic Failover (PAF) zodanig geconfigureerd worden dat er failover en monitoring gedaan kan worden.


%----------------------------------------------------------------------------------------
%	FORTHCOMING RESEARCH
%----------------------------------------------------------------------------------------
\color{HoGentAccent1} 
\section*{Toekomstig onderzoek}
\color{black}
Bijkomend onderzoek zal zeker interessant zijn voor in de toekomst. Hoe wordt High Availability binnen tien jaar geïmplementeerd in PostgreSQL clusters? Welke nieuwe noden in een cluster zijn er dan van belang? Vervolgonderzoek zal zeker een meerwaarde bieden aan bedrijven die belang hechten aan de duurzaamheid van hun clusters.
Dit omdat de specifieke (bedrijfs)noden van High Availability kunnen veranderen en omdat er binnen tien jaar een nieuwe PostgreSQL High Availability cluster oplossing kan aanwezig zijn die misschien beter is.


%----------------------------------------------------------------------------------------

\end{multicols}
\end{document}